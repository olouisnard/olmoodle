\documentclass[12pt]{article}

\usepackage[a4paper,landscape]{geometry}

\usepackage{moodle}

\usepackage{amsmath}
\usepackage{amssymb}
\usepackage{amsfonts}
\usepackage{amsthm,thmtools}
\usepackage{graphicx}

\usepackage{layouts}


\usepackage[utf8]{inputenc}
\usepackage[cyr]{aeguill}
\usepackage{xspace}


\usepackage[french]{babel}

\usepackage{ifthen}
\usepackage{verbatim} % Pour mettre des parties en commentaire

\usepackage{mymaths}
\usepackage{hyperref}



\begin{document}



\begin{quiz}{TRANSFERT DE MATIÈRE/LOUISNARD/NUMERIQUES/FRUITAUTO} 

 \begin{cloze}{Séchage d un fruit} 
Un fruit sphérique de diamètre $D$, humide en surface, est séché par de l'air à la température $T_\infty$, d'humidité relative $\psi$, s'écoulant à la vitesse $U_\infty$.

On veut calculer la vitesse d'évaporation de la pellicule d'eau en surface.

 

Les données sont les suivantes :

 

Diamètre $D = 5.0\,  \mathrm{cm} $

Température $T = 42.0\,  \mathrm{^\circ\mathrm{C}} $

Humidité relative $\psi = 51.\, \% $

Vitesse $U_\infty = 0.0\,  \mathrm{m}\,  \mathrm{s}^{-1} $

Pression atmosphérique $p_{\text{atm}} = 101300.\,  \mathrm{Pa} $

Pression de valeur saturante de l’eau à la température donnée $p_{\text{sat}}(T) = 0.\,  \mathrm{Pa} $

Coefficient de diffusion air/vapeur d’eau $D_{AV} = 0.0\,  \mathrm{m}^{2}\,  \mathrm{s}^{-1} $

Viscosité cinématique de l’air $\nu = 0.0\,  \mathrm{m}^{2}\,  \mathrm{s}^{-1} $

Ces données vous sont personnelles.

 

Calculez les grandeurs ci-dessous. La corrélation utilisée doit \textbf{faire intervenir le nombre de Schmidt}.

Dans vos réponses, utilisez les notations scientifiques si besoin est ($6.34\, 10^{-5}$ s'écrit 6.34e-5 et $10^{3}$ s'écrit 1e3).

Vous avez droit à une marge d'erreur relative de $5.0\, \% $

Des fourchettes indicatives vous sont fournies en face de chaque réponse. Ce ne sont que des ordres de grandeur pour vous inciter à vérifier vos calculs avant de valider vos réponses, et la bonne réponse peut sortir légèrement de l'intervalle indiqué.

 

Nombre de Reynolds : $\text{Re} =  $
\begin{numerical}[points=1] 
\item[tolerance={0.0000000000e+00}] 0.0000000000e+00 
\end{numerical} 
 $\,$ 
 $ \quad (0. \, \rightarrow \, 0.) $ 

Nombre de Schmidt : $\text{Sc} =  $
\begin{numerical}[points=1] 
\item[tolerance={0.0000000000e+00}] 0.0000000000e+00 
\end{numerical} 
 $\,$ 
 $ \quad (0.0 \, \rightarrow \, 0.0) $ 

Nombre de Sherwood : $\text{Sh} =  $
\begin{numerical}[points=2] 
\item[tolerance={0.0000000000e+00}] 0.0000000000e+00 
\end{numerical} 
 $\,$ 
 $ \quad (0. \, \rightarrow \, 0.) $ 

Coefficient d'échange : $k_m =  $
\begin{numerical}[points=1] 
\item[tolerance={0.0000000000e+00}] 0.0000000000e+00 
\end{numerical} 
 $\,  \mathrm{m}\,  \mathrm{s}^{-1}$ 
 $ \quad (0.0 \, \rightarrow \, 0.0) $ 

Concentration en vapeur à la surface : $C_{V, \text{surface}} =  $
\begin{numerical}[points=2] 
\item[tolerance={0.0000000000e+00}] 0.0000000000e+00 
\end{numerical} 
 $\,  \mathrm{mol}\,  \mathrm{m}^{-3}$ 
 $ \quad (0.00 \, \rightarrow \, 0.00) $ 

Concentration en vapeur incidente : $C_{V, \infty} =  $
\begin{numerical}[points=2] 
\item[tolerance={0.0000000000e+00}] 0.0000000000e+00 
\end{numerical} 
 $\,  \mathrm{mol}\,  \mathrm{m}^{-3}$ 
 $ \quad (0.00 \, \rightarrow \, 0.00) $ 

Flux massique d'évaporation : $\dot{m}_V =  $
\begin{numerical}[points=1] 
\item[tolerance={0.0000000000e+00}] 0.0000000000e+00 
\end{numerical} 
 $\,  \mathrm{kg}\,  \mathrm{s}^{-1}$ 
 $ \quad (0.0 \, \rightarrow \, 0.0) $ 

 

On donne l'épaisseur du film d'eau à la surface du fruit $e = 1.0\,  \mathrm{mm} $

Calculez le temps total d'évaporation : $\tau =  $
\begin{numerical}[points=2] 
\item[tolerance={0.0000000000e+00}] 0.0000000000e+00 
\end{numerical} 
 $\,  \mathrm{s}$ 
 $ \quad (0. \, \rightarrow \, 0.) $ 

\end{cloze} 


 \begin{cloze}{Séchage d un fruit} 
Un fruit sphérique de diamètre $D$, humide en surface, est séché par de l'air à la température $T_\infty$, d'humidité relative $\psi$, s'écoulant à la vitesse $U_\infty$.

On veut calculer la vitesse d'évaporation de la pellicule d'eau en surface.

 

Les données sont les suivantes :

 

Diamètre $D = 5.0\,  \mathrm{cm} $

Température $T = 34.0\,  \mathrm{^\circ\mathrm{C}} $

Humidité relative $\psi = 50.\, \% $

Vitesse $U_\infty = 0.0\,  \mathrm{m}\,  \mathrm{s}^{-1} $

Pression atmosphérique $p_{\text{atm}} = 101300.\,  \mathrm{Pa} $

Pression de valeur saturante de l’eau à la température donnée $p_{\text{sat}}(T) = 0.\,  \mathrm{Pa} $

Coefficient de diffusion air/vapeur d’eau $D_{AV} = 0.0\,  \mathrm{m}^{2}\,  \mathrm{s}^{-1} $

Viscosité cinématique de l’air $\nu = 0.0\,  \mathrm{m}^{2}\,  \mathrm{s}^{-1} $

Ces données vous sont personnelles.

 

Calculez les grandeurs ci-dessous. La corrélation utilisée doit \textbf{faire intervenir le nombre de Schmidt}.

Dans vos réponses, utilisez les notations scientifiques si besoin est ($6.34\, 10^{-5}$ s'écrit 6.34e-5 et $10^{3}$ s'écrit 1e3).

Vous avez droit à une marge d'erreur relative de $5.0\, \% $

Des fourchettes indicatives vous sont fournies en face de chaque réponse. Ce ne sont que des ordres de grandeur pour vous inciter à vérifier vos calculs avant de valider vos réponses, et la bonne réponse peut sortir légèrement de l'intervalle indiqué.

 

Nombre de Reynolds : $\text{Re} =  $
\begin{numerical}[points=1] 
\item[tolerance={0.0000000000e+00}] 0.0000000000e+00 
\end{numerical} 
 $\,$ 
 $ \quad (0. \, \rightarrow \, 0.) $ 

Nombre de Schmidt : $\text{Sc} =  $
\begin{numerical}[points=1] 
\item[tolerance={0.0000000000e+00}] 0.0000000000e+00 
\end{numerical} 
 $\,$ 
 $ \quad (0.0 \, \rightarrow \, 0.0) $ 

Nombre de Sherwood : $\text{Sh} =  $
\begin{numerical}[points=2] 
\item[tolerance={0.0000000000e+00}] 0.0000000000e+00 
\end{numerical} 
 $\,$ 
 $ \quad (0. \, \rightarrow \, 0.) $ 

Coefficient d'échange : $k_m =  $
\begin{numerical}[points=1] 
\item[tolerance={0.0000000000e+00}] 0.0000000000e+00 
\end{numerical} 
 $\,  \mathrm{m}\,  \mathrm{s}^{-1}$ 
 $ \quad (0.0 \, \rightarrow \, 0.0) $ 

Concentration en vapeur à la surface : $C_{V, \text{surface}} =  $
\begin{numerical}[points=2] 
\item[tolerance={0.0000000000e+00}] 0.0000000000e+00 
\end{numerical} 
 $\,  \mathrm{mol}\,  \mathrm{m}^{-3}$ 
 $ \quad (0.00 \, \rightarrow \, 0.00) $ 

Concentration en vapeur incidente : $C_{V, \infty} =  $
\begin{numerical}[points=2] 
\item[tolerance={0.0000000000e+00}] 0.0000000000e+00 
\end{numerical} 
 $\,  \mathrm{mol}\,  \mathrm{m}^{-3}$ 
 $ \quad (0.00 \, \rightarrow \, 0.00) $ 

Flux massique d'évaporation : $\dot{m}_V =  $
\begin{numerical}[points=1] 
\item[tolerance={0.0000000000e+00}] 0.0000000000e+00 
\end{numerical} 
 $\,  \mathrm{kg}\,  \mathrm{s}^{-1}$ 
 $ \quad (0.0 \, \rightarrow \, 0.0) $ 

 

On donne l'épaisseur du film d'eau à la surface du fruit $e = 1.0\,  \mathrm{mm} $

Calculez le temps total d'évaporation : $\tau =  $
\begin{numerical}[points=2] 
\item[tolerance={0.0000000000e+00}] 0.0000000000e+00 
\end{numerical} 
 $\,  \mathrm{s}$ 
 $ \quad (0. \, \rightarrow \, 0.) $ 

\end{cloze} 


 \begin{cloze}{Séchage d un fruit} 
Un fruit sphérique de diamètre $D$, humide en surface, est séché par de l'air à la température $T_\infty$, d'humidité relative $\psi$, s'écoulant à la vitesse $U_\infty$.

On veut calculer la vitesse d'évaporation de la pellicule d'eau en surface.

 

Les données sont les suivantes :

 

Diamètre $D = 4.0\,  \mathrm{cm} $

Température $T = 27.0\,  \mathrm{^\circ\mathrm{C}} $

Humidité relative $\psi = 36.\, \% $

Vitesse $U_\infty = 0.0\,  \mathrm{m}\,  \mathrm{s}^{-1} $

Pression atmosphérique $p_{\text{atm}} = 101300.\,  \mathrm{Pa} $

Pression de valeur saturante de l’eau à la température donnée $p_{\text{sat}}(T) = 0.\,  \mathrm{Pa} $

Coefficient de diffusion air/vapeur d’eau $D_{AV} = 0.0\,  \mathrm{m}^{2}\,  \mathrm{s}^{-1} $

Viscosité cinématique de l’air $\nu = 0.0\,  \mathrm{m}^{2}\,  \mathrm{s}^{-1} $

Ces données vous sont personnelles.

 

Calculez les grandeurs ci-dessous. La corrélation utilisée doit \textbf{faire intervenir le nombre de Schmidt}.

Dans vos réponses, utilisez les notations scientifiques si besoin est ($6.34\, 10^{-5}$ s'écrit 6.34e-5 et $10^{3}$ s'écrit 1e3).

Vous avez droit à une marge d'erreur relative de $5.0\, \% $

Des fourchettes indicatives vous sont fournies en face de chaque réponse. Ce ne sont que des ordres de grandeur pour vous inciter à vérifier vos calculs avant de valider vos réponses, et la bonne réponse peut sortir légèrement de l'intervalle indiqué.

 

Nombre de Reynolds : $\text{Re} =  $
\begin{numerical}[points=1] 
\item[tolerance={0.0000000000e+00}] 0.0000000000e+00 
\end{numerical} 
 $\,$ 
 $ \quad (0. \, \rightarrow \, 0.) $ 

Nombre de Schmidt : $\text{Sc} =  $
\begin{numerical}[points=1] 
\item[tolerance={0.0000000000e+00}] 0.0000000000e+00 
\end{numerical} 
 $\,$ 
 $ \quad (0.0 \, \rightarrow \, 0.0) $ 

Nombre de Sherwood : $\text{Sh} =  $
\begin{numerical}[points=2] 
\item[tolerance={0.0000000000e+00}] 0.0000000000e+00 
\end{numerical} 
 $\,$ 
 $ \quad (0. \, \rightarrow \, 0.) $ 

Coefficient d'échange : $k_m =  $
\begin{numerical}[points=1] 
\item[tolerance={0.0000000000e+00}] 0.0000000000e+00 
\end{numerical} 
 $\,  \mathrm{m}\,  \mathrm{s}^{-1}$ 
 $ \quad (0.0 \, \rightarrow \, 0.0) $ 

Concentration en vapeur à la surface : $C_{V, \text{surface}} =  $
\begin{numerical}[points=2] 
\item[tolerance={0.0000000000e+00}] 0.0000000000e+00 
\end{numerical} 
 $\,  \mathrm{mol}\,  \mathrm{m}^{-3}$ 
 $ \quad (0.00 \, \rightarrow \, 0.00) $ 

Concentration en vapeur incidente : $C_{V, \infty} =  $
\begin{numerical}[points=2] 
\item[tolerance={0.0000000000e+00}] 0.0000000000e+00 
\end{numerical} 
 $\,  \mathrm{mol}\,  \mathrm{m}^{-3}$ 
 $ \quad (0.00 \, \rightarrow \, 0.00) $ 

Flux massique d'évaporation : $\dot{m}_V =  $
\begin{numerical}[points=1] 
\item[tolerance={0.0000000000e+00}] 0.0000000000e+00 
\end{numerical} 
 $\,  \mathrm{kg}\,  \mathrm{s}^{-1}$ 
 $ \quad (0.0 \, \rightarrow \, 0.0) $ 

 

On donne l'épaisseur du film d'eau à la surface du fruit $e = 1.0\,  \mathrm{mm} $

Calculez le temps total d'évaporation : $\tau =  $
\begin{numerical}[points=2] 
\item[tolerance={0.0000000000e+00}] 0.0000000000e+00 
\end{numerical} 
 $\,  \mathrm{s}$ 
 $ \quad (0. \, \rightarrow \, 0.) $ 

\end{cloze} 


 \begin{cloze}{Séchage d un fruit} 
Un fruit sphérique de diamètre $D$, humide en surface, est séché par de l'air à la température $T_\infty$, d'humidité relative $\psi$, s'écoulant à la vitesse $U_\infty$.

On veut calculer la vitesse d'évaporation de la pellicule d'eau en surface.

 

Les données sont les suivantes :

 

Diamètre $D = 6.0\,  \mathrm{cm} $

Température $T = 30.0\,  \mathrm{^\circ\mathrm{C}} $

Humidité relative $\psi = 32.\, \% $

Vitesse $U_\infty = 0.0\,  \mathrm{m}\,  \mathrm{s}^{-1} $

Pression atmosphérique $p_{\text{atm}} = 101300.\,  \mathrm{Pa} $

Pression de valeur saturante de l’eau à la température donnée $p_{\text{sat}}(T) = 0.\,  \mathrm{Pa} $

Coefficient de diffusion air/vapeur d’eau $D_{AV} = 0.0\,  \mathrm{m}^{2}\,  \mathrm{s}^{-1} $

Viscosité cinématique de l’air $\nu = 0.0\,  \mathrm{m}^{2}\,  \mathrm{s}^{-1} $

Ces données vous sont personnelles.

 

Calculez les grandeurs ci-dessous. La corrélation utilisée doit \textbf{faire intervenir le nombre de Schmidt}.

Dans vos réponses, utilisez les notations scientifiques si besoin est ($6.34\, 10^{-5}$ s'écrit 6.34e-5 et $10^{3}$ s'écrit 1e3).

Vous avez droit à une marge d'erreur relative de $5.0\, \% $

Des fourchettes indicatives vous sont fournies en face de chaque réponse. Ce ne sont que des ordres de grandeur pour vous inciter à vérifier vos calculs avant de valider vos réponses, et la bonne réponse peut sortir légèrement de l'intervalle indiqué.

 

Nombre de Reynolds : $\text{Re} =  $
\begin{numerical}[points=1] 
\item[tolerance={0.0000000000e+00}] 0.0000000000e+00 
\end{numerical} 
 $\,$ 
 $ \quad (0. \, \rightarrow \, 0.) $ 

Nombre de Schmidt : $\text{Sc} =  $
\begin{numerical}[points=1] 
\item[tolerance={0.0000000000e+00}] 0.0000000000e+00 
\end{numerical} 
 $\,$ 
 $ \quad (0.0 \, \rightarrow \, 0.0) $ 

Nombre de Sherwood : $\text{Sh} =  $
\begin{numerical}[points=2] 
\item[tolerance={0.0000000000e+00}] 0.0000000000e+00 
\end{numerical} 
 $\,$ 
 $ \quad (0. \, \rightarrow \, 0.) $ 

Coefficient d'échange : $k_m =  $
\begin{numerical}[points=1] 
\item[tolerance={0.0000000000e+00}] 0.0000000000e+00 
\end{numerical} 
 $\,  \mathrm{m}\,  \mathrm{s}^{-1}$ 
 $ \quad (0.0 \, \rightarrow \, 0.0) $ 

Concentration en vapeur à la surface : $C_{V, \text{surface}} =  $
\begin{numerical}[points=2] 
\item[tolerance={0.0000000000e+00}] 0.0000000000e+00 
\end{numerical} 
 $\,  \mathrm{mol}\,  \mathrm{m}^{-3}$ 
 $ \quad (0.00 \, \rightarrow \, 0.00) $ 

Concentration en vapeur incidente : $C_{V, \infty} =  $
\begin{numerical}[points=2] 
\item[tolerance={0.0000000000e+00}] 0.0000000000e+00 
\end{numerical} 
 $\,  \mathrm{mol}\,  \mathrm{m}^{-3}$ 
 $ \quad (0.00 \, \rightarrow \, 0.00) $ 

Flux massique d'évaporation : $\dot{m}_V =  $
\begin{numerical}[points=1] 
\item[tolerance={0.0000000000e+00}] 0.0000000000e+00 
\end{numerical} 
 $\,  \mathrm{kg}\,  \mathrm{s}^{-1}$ 
 $ \quad (0.0 \, \rightarrow \, 0.0) $ 

 

On donne l'épaisseur du film d'eau à la surface du fruit $e = 1.0\,  \mathrm{mm} $

Calculez le temps total d'évaporation : $\tau =  $
\begin{numerical}[points=2] 
\item[tolerance={0.0000000000e+00}] 0.0000000000e+00 
\end{numerical} 
 $\,  \mathrm{s}$ 
 $ \quad (0. \, \rightarrow \, 0.) $ 

\end{cloze} 


 \begin{cloze}{Séchage d un fruit} 
Un fruit sphérique de diamètre $D$, humide en surface, est séché par de l'air à la température $T_\infty$, d'humidité relative $\psi$, s'écoulant à la vitesse $U_\infty$.

On veut calculer la vitesse d'évaporation de la pellicule d'eau en surface.

 

Les données sont les suivantes :

 

Diamètre $D = 5.0\,  \mathrm{cm} $

Température $T = 31.0\,  \mathrm{^\circ\mathrm{C}} $

Humidité relative $\psi = 38.\, \% $

Vitesse $U_\infty = 0.0\,  \mathrm{m}\,  \mathrm{s}^{-1} $

Pression atmosphérique $p_{\text{atm}} = 101300.\,  \mathrm{Pa} $

Pression de valeur saturante de l’eau à la température donnée $p_{\text{sat}}(T) = 0.\,  \mathrm{Pa} $

Coefficient de diffusion air/vapeur d’eau $D_{AV} = 0.0\,  \mathrm{m}^{2}\,  \mathrm{s}^{-1} $

Viscosité cinématique de l’air $\nu = 0.0\,  \mathrm{m}^{2}\,  \mathrm{s}^{-1} $

Ces données vous sont personnelles.

 

Calculez les grandeurs ci-dessous. La corrélation utilisée doit \textbf{faire intervenir le nombre de Schmidt}.

Dans vos réponses, utilisez les notations scientifiques si besoin est ($6.34\, 10^{-5}$ s'écrit 6.34e-5 et $10^{3}$ s'écrit 1e3).

Vous avez droit à une marge d'erreur relative de $5.0\, \% $

Des fourchettes indicatives vous sont fournies en face de chaque réponse. Ce ne sont que des ordres de grandeur pour vous inciter à vérifier vos calculs avant de valider vos réponses, et la bonne réponse peut sortir légèrement de l'intervalle indiqué.

 

Nombre de Reynolds : $\text{Re} =  $
\begin{numerical}[points=1] 
\item[tolerance={0.0000000000e+00}] 0.0000000000e+00 
\end{numerical} 
 $\,$ 
 $ \quad (0. \, \rightarrow \, 0.) $ 

Nombre de Schmidt : $\text{Sc} =  $
\begin{numerical}[points=1] 
\item[tolerance={0.0000000000e+00}] 0.0000000000e+00 
\end{numerical} 
 $\,$ 
 $ \quad (0.0 \, \rightarrow \, 0.0) $ 

Nombre de Sherwood : $\text{Sh} =  $
\begin{numerical}[points=2] 
\item[tolerance={0.0000000000e+00}] 0.0000000000e+00 
\end{numerical} 
 $\,$ 
 $ \quad (0. \, \rightarrow \, 0.) $ 

Coefficient d'échange : $k_m =  $
\begin{numerical}[points=1] 
\item[tolerance={0.0000000000e+00}] 0.0000000000e+00 
\end{numerical} 
 $\,  \mathrm{m}\,  \mathrm{s}^{-1}$ 
 $ \quad (0.0 \, \rightarrow \, 0.0) $ 

Concentration en vapeur à la surface : $C_{V, \text{surface}} =  $
\begin{numerical}[points=2] 
\item[tolerance={0.0000000000e+00}] 0.0000000000e+00 
\end{numerical} 
 $\,  \mathrm{mol}\,  \mathrm{m}^{-3}$ 
 $ \quad (0.00 \, \rightarrow \, 0.00) $ 

Concentration en vapeur incidente : $C_{V, \infty} =  $
\begin{numerical}[points=2] 
\item[tolerance={0.0000000000e+00}] 0.0000000000e+00 
\end{numerical} 
 $\,  \mathrm{mol}\,  \mathrm{m}^{-3}$ 
 $ \quad (0.00 \, \rightarrow \, 0.00) $ 

Flux massique d'évaporation : $\dot{m}_V =  $
\begin{numerical}[points=1] 
\item[tolerance={0.0000000000e+00}] 0.0000000000e+00 
\end{numerical} 
 $\,  \mathrm{kg}\,  \mathrm{s}^{-1}$ 
 $ \quad (0.0 \, \rightarrow \, 0.0) $ 

 

On donne l'épaisseur du film d'eau à la surface du fruit $e = 1.0\,  \mathrm{mm} $

Calculez le temps total d'évaporation : $\tau =  $
\begin{numerical}[points=2] 
\item[tolerance={0.0000000000e+00}] 0.0000000000e+00 
\end{numerical} 
 $\,  \mathrm{s}$ 
 $ \quad (0. \, \rightarrow \, 0.) $ 

\end{cloze} 


 \begin{cloze}{Séchage d un fruit} 
Un fruit sphérique de diamètre $D$, humide en surface, est séché par de l'air à la température $T_\infty$, d'humidité relative $\psi$, s'écoulant à la vitesse $U_\infty$.

On veut calculer la vitesse d'évaporation de la pellicule d'eau en surface.

 

Les données sont les suivantes :

 

Diamètre $D = 9.0\,  \mathrm{cm} $

Température $T = 30.0\,  \mathrm{^\circ\mathrm{C}} $

Humidité relative $\psi = 56.\, \% $

Vitesse $U_\infty = 0.0\,  \mathrm{m}\,  \mathrm{s}^{-1} $

Pression atmosphérique $p_{\text{atm}} = 101300.\,  \mathrm{Pa} $

Pression de valeur saturante de l’eau à la température donnée $p_{\text{sat}}(T) = 0.\,  \mathrm{Pa} $

Coefficient de diffusion air/vapeur d’eau $D_{AV} = 0.0\,  \mathrm{m}^{2}\,  \mathrm{s}^{-1} $

Viscosité cinématique de l’air $\nu = 0.0\,  \mathrm{m}^{2}\,  \mathrm{s}^{-1} $

Ces données vous sont personnelles.

 

Calculez les grandeurs ci-dessous. La corrélation utilisée doit \textbf{faire intervenir le nombre de Schmidt}.

Dans vos réponses, utilisez les notations scientifiques si besoin est ($6.34\, 10^{-5}$ s'écrit 6.34e-5 et $10^{3}$ s'écrit 1e3).

Vous avez droit à une marge d'erreur relative de $5.0\, \% $

Des fourchettes indicatives vous sont fournies en face de chaque réponse. Ce ne sont que des ordres de grandeur pour vous inciter à vérifier vos calculs avant de valider vos réponses, et la bonne réponse peut sortir légèrement de l'intervalle indiqué.

 

Nombre de Reynolds : $\text{Re} =  $
\begin{numerical}[points=1] 
\item[tolerance={0.0000000000e+00}] 0.0000000000e+00 
\end{numerical} 
 $\,$ 
 $ \quad (0. \, \rightarrow \, 0.) $ 

Nombre de Schmidt : $\text{Sc} =  $
\begin{numerical}[points=1] 
\item[tolerance={0.0000000000e+00}] 0.0000000000e+00 
\end{numerical} 
 $\,$ 
 $ \quad (0.0 \, \rightarrow \, 0.0) $ 

Nombre de Sherwood : $\text{Sh} =  $
\begin{numerical}[points=2] 
\item[tolerance={0.0000000000e+00}] 0.0000000000e+00 
\end{numerical} 
 $\,$ 
 $ \quad (0. \, \rightarrow \, 0.) $ 

Coefficient d'échange : $k_m =  $
\begin{numerical}[points=1] 
\item[tolerance={0.0000000000e+00}] 0.0000000000e+00 
\end{numerical} 
 $\,  \mathrm{m}\,  \mathrm{s}^{-1}$ 
 $ \quad (0.0 \, \rightarrow \, 0.0) $ 

Concentration en vapeur à la surface : $C_{V, \text{surface}} =  $
\begin{numerical}[points=2] 
\item[tolerance={0.0000000000e+00}] 0.0000000000e+00 
\end{numerical} 
 $\,  \mathrm{mol}\,  \mathrm{m}^{-3}$ 
 $ \quad (0.00 \, \rightarrow \, 0.00) $ 

Concentration en vapeur incidente : $C_{V, \infty} =  $
\begin{numerical}[points=2] 
\item[tolerance={0.0000000000e+00}] 0.0000000000e+00 
\end{numerical} 
 $\,  \mathrm{mol}\,  \mathrm{m}^{-3}$ 
 $ \quad (0.00 \, \rightarrow \, 0.00) $ 

Flux massique d'évaporation : $\dot{m}_V =  $
\begin{numerical}[points=1] 
\item[tolerance={0.0000000000e+00}] 0.0000000000e+00 
\end{numerical} 
 $\,  \mathrm{kg}\,  \mathrm{s}^{-1}$ 
 $ \quad (0.0 \, \rightarrow \, 0.0) $ 

 

On donne l'épaisseur du film d'eau à la surface du fruit $e = 1.0\,  \mathrm{mm} $

Calculez le temps total d'évaporation : $\tau =  $
\begin{numerical}[points=2] 
\item[tolerance={0.0000000000e+00}] 0.0000000000e+00 
\end{numerical} 
 $\,  \mathrm{s}$ 
 $ \quad (0. \, \rightarrow \, 0.) $ 

\end{cloze} 


 \begin{cloze}{Séchage d un fruit} 
Un fruit sphérique de diamètre $D$, humide en surface, est séché par de l'air à la température $T_\infty$, d'humidité relative $\psi$, s'écoulant à la vitesse $U_\infty$.

On veut calculer la vitesse d'évaporation de la pellicule d'eau en surface.

 

Les données sont les suivantes :

 

Diamètre $D = 7.0\,  \mathrm{cm} $

Température $T = 42.0\,  \mathrm{^\circ\mathrm{C}} $

Humidité relative $\psi = 46.\, \% $

Vitesse $U_\infty = 0.0\,  \mathrm{m}\,  \mathrm{s}^{-1} $

Pression atmosphérique $p_{\text{atm}} = 101300.\,  \mathrm{Pa} $

Pression de valeur saturante de l’eau à la température donnée $p_{\text{sat}}(T) = 0.\,  \mathrm{Pa} $

Coefficient de diffusion air/vapeur d’eau $D_{AV} = 0.0\,  \mathrm{m}^{2}\,  \mathrm{s}^{-1} $

Viscosité cinématique de l’air $\nu = 0.0\,  \mathrm{m}^{2}\,  \mathrm{s}^{-1} $

Ces données vous sont personnelles.

 

Calculez les grandeurs ci-dessous. La corrélation utilisée doit \textbf{faire intervenir le nombre de Schmidt}.

Dans vos réponses, utilisez les notations scientifiques si besoin est ($6.34\, 10^{-5}$ s'écrit 6.34e-5 et $10^{3}$ s'écrit 1e3).

Vous avez droit à une marge d'erreur relative de $5.0\, \% $

Des fourchettes indicatives vous sont fournies en face de chaque réponse. Ce ne sont que des ordres de grandeur pour vous inciter à vérifier vos calculs avant de valider vos réponses, et la bonne réponse peut sortir légèrement de l'intervalle indiqué.

 

Nombre de Reynolds : $\text{Re} =  $
\begin{numerical}[points=1] 
\item[tolerance={0.0000000000e+00}] 0.0000000000e+00 
\end{numerical} 
 $\,$ 
 $ \quad (0. \, \rightarrow \, 0.) $ 

Nombre de Schmidt : $\text{Sc} =  $
\begin{numerical}[points=1] 
\item[tolerance={0.0000000000e+00}] 0.0000000000e+00 
\end{numerical} 
 $\,$ 
 $ \quad (0.0 \, \rightarrow \, 0.0) $ 

Nombre de Sherwood : $\text{Sh} =  $
\begin{numerical}[points=2] 
\item[tolerance={0.0000000000e+00}] 0.0000000000e+00 
\end{numerical} 
 $\,$ 
 $ \quad (0. \, \rightarrow \, 0.) $ 

Coefficient d'échange : $k_m =  $
\begin{numerical}[points=1] 
\item[tolerance={0.0000000000e+00}] 0.0000000000e+00 
\end{numerical} 
 $\,  \mathrm{m}\,  \mathrm{s}^{-1}$ 
 $ \quad (0.0 \, \rightarrow \, 0.0) $ 

Concentration en vapeur à la surface : $C_{V, \text{surface}} =  $
\begin{numerical}[points=2] 
\item[tolerance={0.0000000000e+00}] 0.0000000000e+00 
\end{numerical} 
 $\,  \mathrm{mol}\,  \mathrm{m}^{-3}$ 
 $ \quad (0.00 \, \rightarrow \, 0.00) $ 

Concentration en vapeur incidente : $C_{V, \infty} =  $
\begin{numerical}[points=2] 
\item[tolerance={0.0000000000e+00}] 0.0000000000e+00 
\end{numerical} 
 $\,  \mathrm{mol}\,  \mathrm{m}^{-3}$ 
 $ \quad (0.00 \, \rightarrow \, 0.00) $ 

Flux massique d'évaporation : $\dot{m}_V =  $
\begin{numerical}[points=1] 
\item[tolerance={0.0000000000e+00}] 0.0000000000e+00 
\end{numerical} 
 $\,  \mathrm{kg}\,  \mathrm{s}^{-1}$ 
 $ \quad (0.0 \, \rightarrow \, 0.0) $ 

 

On donne l'épaisseur du film d'eau à la surface du fruit $e = 1.0\,  \mathrm{mm} $

Calculez le temps total d'évaporation : $\tau =  $
\begin{numerical}[points=2] 
\item[tolerance={0.0000000000e+00}] 0.0000000000e+00 
\end{numerical} 
 $\,  \mathrm{s}$ 
 $ \quad (0. \, \rightarrow \, 0.) $ 

\end{cloze} 


 \begin{cloze}{Séchage d un fruit} 
Un fruit sphérique de diamètre $D$, humide en surface, est séché par de l'air à la température $T_\infty$, d'humidité relative $\psi$, s'écoulant à la vitesse $U_\infty$.

On veut calculer la vitesse d'évaporation de la pellicule d'eau en surface.

 

Les données sont les suivantes :

 

Diamètre $D = 9.0\,  \mathrm{cm} $

Température $T = 43.0\,  \mathrm{^\circ\mathrm{C}} $

Humidité relative $\psi = 43.\, \% $

Vitesse $U_\infty = 0.0\,  \mathrm{m}\,  \mathrm{s}^{-1} $

Pression atmosphérique $p_{\text{atm}} = 101300.\,  \mathrm{Pa} $

Pression de valeur saturante de l’eau à la température donnée $p_{\text{sat}}(T) = 0.\,  \mathrm{Pa} $

Coefficient de diffusion air/vapeur d’eau $D_{AV} = 0.0\,  \mathrm{m}^{2}\,  \mathrm{s}^{-1} $

Viscosité cinématique de l’air $\nu = 0.0\,  \mathrm{m}^{2}\,  \mathrm{s}^{-1} $

Ces données vous sont personnelles.

 

Calculez les grandeurs ci-dessous. La corrélation utilisée doit \textbf{faire intervenir le nombre de Schmidt}.

Dans vos réponses, utilisez les notations scientifiques si besoin est ($6.34\, 10^{-5}$ s'écrit 6.34e-5 et $10^{3}$ s'écrit 1e3).

Vous avez droit à une marge d'erreur relative de $5.0\, \% $

Des fourchettes indicatives vous sont fournies en face de chaque réponse. Ce ne sont que des ordres de grandeur pour vous inciter à vérifier vos calculs avant de valider vos réponses, et la bonne réponse peut sortir légèrement de l'intervalle indiqué.

 

Nombre de Reynolds : $\text{Re} =  $
\begin{numerical}[points=1] 
\item[tolerance={0.0000000000e+00}] 0.0000000000e+00 
\end{numerical} 
 $\,$ 
 $ \quad (0. \, \rightarrow \, 0.) $ 

Nombre de Schmidt : $\text{Sc} =  $
\begin{numerical}[points=1] 
\item[tolerance={0.0000000000e+00}] 0.0000000000e+00 
\end{numerical} 
 $\,$ 
 $ \quad (0.0 \, \rightarrow \, 0.0) $ 

Nombre de Sherwood : $\text{Sh} =  $
\begin{numerical}[points=2] 
\item[tolerance={0.0000000000e+00}] 0.0000000000e+00 
\end{numerical} 
 $\,$ 
 $ \quad (0. \, \rightarrow \, 0.) $ 

Coefficient d'échange : $k_m =  $
\begin{numerical}[points=1] 
\item[tolerance={0.0000000000e+00}] 0.0000000000e+00 
\end{numerical} 
 $\,  \mathrm{m}\,  \mathrm{s}^{-1}$ 
 $ \quad (0.0 \, \rightarrow \, 0.0) $ 

Concentration en vapeur à la surface : $C_{V, \text{surface}} =  $
\begin{numerical}[points=2] 
\item[tolerance={0.0000000000e+00}] 0.0000000000e+00 
\end{numerical} 
 $\,  \mathrm{mol}\,  \mathrm{m}^{-3}$ 
 $ \quad (0.00 \, \rightarrow \, 0.00) $ 

Concentration en vapeur incidente : $C_{V, \infty} =  $
\begin{numerical}[points=2] 
\item[tolerance={0.0000000000e+00}] 0.0000000000e+00 
\end{numerical} 
 $\,  \mathrm{mol}\,  \mathrm{m}^{-3}$ 
 $ \quad (0.00 \, \rightarrow \, 0.00) $ 

Flux massique d'évaporation : $\dot{m}_V =  $
\begin{numerical}[points=1] 
\item[tolerance={0.0000000000e+00}] 0.0000000000e+00 
\end{numerical} 
 $\,  \mathrm{kg}\,  \mathrm{s}^{-1}$ 
 $ \quad (0.0 \, \rightarrow \, 0.0) $ 

 

On donne l'épaisseur du film d'eau à la surface du fruit $e = 1.0\,  \mathrm{mm} $

Calculez le temps total d'évaporation : $\tau =  $
\begin{numerical}[points=2] 
\item[tolerance={0.0000000000e+00}] 0.0000000000e+00 
\end{numerical} 
 $\,  \mathrm{s}$ 
 $ \quad (0. \, \rightarrow \, 0.) $ 

\end{cloze} 


 \begin{cloze}{Séchage d un fruit} 
Un fruit sphérique de diamètre $D$, humide en surface, est séché par de l'air à la température $T_\infty$, d'humidité relative $\psi$, s'écoulant à la vitesse $U_\infty$.

On veut calculer la vitesse d'évaporation de la pellicule d'eau en surface.

 

Les données sont les suivantes :

 

Diamètre $D = 9.0\,  \mathrm{cm} $

Température $T = 38.0\,  \mathrm{^\circ\mathrm{C}} $

Humidité relative $\psi = 42.\, \% $

Vitesse $U_\infty = 0.0\,  \mathrm{m}\,  \mathrm{s}^{-1} $

Pression atmosphérique $p_{\text{atm}} = 101300.\,  \mathrm{Pa} $

Pression de valeur saturante de l’eau à la température donnée $p_{\text{sat}}(T) = 0.\,  \mathrm{Pa} $

Coefficient de diffusion air/vapeur d’eau $D_{AV} = 0.0\,  \mathrm{m}^{2}\,  \mathrm{s}^{-1} $

Viscosité cinématique de l’air $\nu = 0.0\,  \mathrm{m}^{2}\,  \mathrm{s}^{-1} $

Ces données vous sont personnelles.

 

Calculez les grandeurs ci-dessous. La corrélation utilisée doit \textbf{faire intervenir le nombre de Schmidt}.

Dans vos réponses, utilisez les notations scientifiques si besoin est ($6.34\, 10^{-5}$ s'écrit 6.34e-5 et $10^{3}$ s'écrit 1e3).

Vous avez droit à une marge d'erreur relative de $5.0\, \% $

Des fourchettes indicatives vous sont fournies en face de chaque réponse. Ce ne sont que des ordres de grandeur pour vous inciter à vérifier vos calculs avant de valider vos réponses, et la bonne réponse peut sortir légèrement de l'intervalle indiqué.

 

Nombre de Reynolds : $\text{Re} =  $
\begin{numerical}[points=1] 
\item[tolerance={0.0000000000e+00}] 0.0000000000e+00 
\end{numerical} 
 $\,$ 
 $ \quad (0. \, \rightarrow \, 0.) $ 

Nombre de Schmidt : $\text{Sc} =  $
\begin{numerical}[points=1] 
\item[tolerance={0.0000000000e+00}] 0.0000000000e+00 
\end{numerical} 
 $\,$ 
 $ \quad (0.0 \, \rightarrow \, 0.0) $ 

Nombre de Sherwood : $\text{Sh} =  $
\begin{numerical}[points=2] 
\item[tolerance={0.0000000000e+00}] 0.0000000000e+00 
\end{numerical} 
 $\,$ 
 $ \quad (0. \, \rightarrow \, 0.) $ 

Coefficient d'échange : $k_m =  $
\begin{numerical}[points=1] 
\item[tolerance={0.0000000000e+00}] 0.0000000000e+00 
\end{numerical} 
 $\,  \mathrm{m}\,  \mathrm{s}^{-1}$ 
 $ \quad (0.0 \, \rightarrow \, 0.0) $ 

Concentration en vapeur à la surface : $C_{V, \text{surface}} =  $
\begin{numerical}[points=2] 
\item[tolerance={0.0000000000e+00}] 0.0000000000e+00 
\end{numerical} 
 $\,  \mathrm{mol}\,  \mathrm{m}^{-3}$ 
 $ \quad (0.00 \, \rightarrow \, 0.00) $ 

Concentration en vapeur incidente : $C_{V, \infty} =  $
\begin{numerical}[points=2] 
\item[tolerance={0.0000000000e+00}] 0.0000000000e+00 
\end{numerical} 
 $\,  \mathrm{mol}\,  \mathrm{m}^{-3}$ 
 $ \quad (0.00 \, \rightarrow \, 0.00) $ 

Flux massique d'évaporation : $\dot{m}_V =  $
\begin{numerical}[points=1] 
\item[tolerance={0.0000000000e+00}] 0.0000000000e+00 
\end{numerical} 
 $\,  \mathrm{kg}\,  \mathrm{s}^{-1}$ 
 $ \quad (0.0 \, \rightarrow \, 0.0) $ 

 

On donne l'épaisseur du film d'eau à la surface du fruit $e = 1.0\,  \mathrm{mm} $

Calculez le temps total d'évaporation : $\tau =  $
\begin{numerical}[points=2] 
\item[tolerance={0.0000000000e+00}] 0.0000000000e+00 
\end{numerical} 
 $\,  \mathrm{s}$ 
 $ \quad (0. \, \rightarrow \, 0.) $ 

\end{cloze} 


 \begin{cloze}{Séchage d un fruit} 
Un fruit sphérique de diamètre $D$, humide en surface, est séché par de l'air à la température $T_\infty$, d'humidité relative $\psi$, s'écoulant à la vitesse $U_\infty$.

On veut calculer la vitesse d'évaporation de la pellicule d'eau en surface.

 

Les données sont les suivantes :

 

Diamètre $D = 7.0\,  \mathrm{cm} $

Température $T = 38.0\,  \mathrm{^\circ\mathrm{C}} $

Humidité relative $\psi = 53.\, \% $

Vitesse $U_\infty = 0.0\,  \mathrm{m}\,  \mathrm{s}^{-1} $

Pression atmosphérique $p_{\text{atm}} = 101300.\,  \mathrm{Pa} $

Pression de valeur saturante de l’eau à la température donnée $p_{\text{sat}}(T) = 0.\,  \mathrm{Pa} $

Coefficient de diffusion air/vapeur d’eau $D_{AV} = 0.0\,  \mathrm{m}^{2}\,  \mathrm{s}^{-1} $

Viscosité cinématique de l’air $\nu = 0.0\,  \mathrm{m}^{2}\,  \mathrm{s}^{-1} $

Ces données vous sont personnelles.

 

Calculez les grandeurs ci-dessous. La corrélation utilisée doit \textbf{faire intervenir le nombre de Schmidt}.

Dans vos réponses, utilisez les notations scientifiques si besoin est ($6.34\, 10^{-5}$ s'écrit 6.34e-5 et $10^{3}$ s'écrit 1e3).

Vous avez droit à une marge d'erreur relative de $5.0\, \% $

Des fourchettes indicatives vous sont fournies en face de chaque réponse. Ce ne sont que des ordres de grandeur pour vous inciter à vérifier vos calculs avant de valider vos réponses, et la bonne réponse peut sortir légèrement de l'intervalle indiqué.

 

Nombre de Reynolds : $\text{Re} =  $
\begin{numerical}[points=1] 
\item[tolerance={0.0000000000e+00}] 0.0000000000e+00 
\end{numerical} 
 $\,$ 
 $ \quad (0. \, \rightarrow \, 0.) $ 

Nombre de Schmidt : $\text{Sc} =  $
\begin{numerical}[points=1] 
\item[tolerance={0.0000000000e+00}] 0.0000000000e+00 
\end{numerical} 
 $\,$ 
 $ \quad (0.0 \, \rightarrow \, 0.0) $ 

Nombre de Sherwood : $\text{Sh} =  $
\begin{numerical}[points=2] 
\item[tolerance={0.0000000000e+00}] 0.0000000000e+00 
\end{numerical} 
 $\,$ 
 $ \quad (0. \, \rightarrow \, 0.) $ 

Coefficient d'échange : $k_m =  $
\begin{numerical}[points=1] 
\item[tolerance={0.0000000000e+00}] 0.0000000000e+00 
\end{numerical} 
 $\,  \mathrm{m}\,  \mathrm{s}^{-1}$ 
 $ \quad (0.0 \, \rightarrow \, 0.0) $ 

Concentration en vapeur à la surface : $C_{V, \text{surface}} =  $
\begin{numerical}[points=2] 
\item[tolerance={0.0000000000e+00}] 0.0000000000e+00 
\end{numerical} 
 $\,  \mathrm{mol}\,  \mathrm{m}^{-3}$ 
 $ \quad (0.00 \, \rightarrow \, 0.00) $ 

Concentration en vapeur incidente : $C_{V, \infty} =  $
\begin{numerical}[points=2] 
\item[tolerance={0.0000000000e+00}] 0.0000000000e+00 
\end{numerical} 
 $\,  \mathrm{mol}\,  \mathrm{m}^{-3}$ 
 $ \quad (0.00 \, \rightarrow \, 0.00) $ 

Flux massique d'évaporation : $\dot{m}_V =  $
\begin{numerical}[points=1] 
\item[tolerance={0.0000000000e+00}] 0.0000000000e+00 
\end{numerical} 
 $\,  \mathrm{kg}\,  \mathrm{s}^{-1}$ 
 $ \quad (0.0 \, \rightarrow \, 0.0) $ 

 

On donne l'épaisseur du film d'eau à la surface du fruit $e = 1.0\,  \mathrm{mm} $

Calculez le temps total d'évaporation : $\tau =  $
\begin{numerical}[points=2] 
\item[tolerance={0.0000000000e+00}] 0.0000000000e+00 
\end{numerical} 
 $\,  \mathrm{s}$ 
 $ \quad (0. \, \rightarrow \, 0.) $ 

\end{cloze} 


 \begin{cloze}{Séchage d un fruit} 
Un fruit sphérique de diamètre $D$, humide en surface, est séché par de l'air à la température $T_\infty$, d'humidité relative $\psi$, s'écoulant à la vitesse $U_\infty$.

On veut calculer la vitesse d'évaporation de la pellicule d'eau en surface.

 

Les données sont les suivantes :

 

Diamètre $D = 4.0\,  \mathrm{cm} $

Température $T = 40.0\,  \mathrm{^\circ\mathrm{C}} $

Humidité relative $\psi = 49.\, \% $

Vitesse $U_\infty = 0.0\,  \mathrm{m}\,  \mathrm{s}^{-1} $

Pression atmosphérique $p_{\text{atm}} = 101300.\,  \mathrm{Pa} $

Pression de valeur saturante de l’eau à la température donnée $p_{\text{sat}}(T) = 0.\,  \mathrm{Pa} $

Coefficient de diffusion air/vapeur d’eau $D_{AV} = 0.0\,  \mathrm{m}^{2}\,  \mathrm{s}^{-1} $

Viscosité cinématique de l’air $\nu = 0.0\,  \mathrm{m}^{2}\,  \mathrm{s}^{-1} $

Ces données vous sont personnelles.

 

Calculez les grandeurs ci-dessous. La corrélation utilisée doit \textbf{faire intervenir le nombre de Schmidt}.

Dans vos réponses, utilisez les notations scientifiques si besoin est ($6.34\, 10^{-5}$ s'écrit 6.34e-5 et $10^{3}$ s'écrit 1e3).

Vous avez droit à une marge d'erreur relative de $5.0\, \% $

Des fourchettes indicatives vous sont fournies en face de chaque réponse. Ce ne sont que des ordres de grandeur pour vous inciter à vérifier vos calculs avant de valider vos réponses, et la bonne réponse peut sortir légèrement de l'intervalle indiqué.

 

Nombre de Reynolds : $\text{Re} =  $
\begin{numerical}[points=1] 
\item[tolerance={0.0000000000e+00}] 0.0000000000e+00 
\end{numerical} 
 $\,$ 
 $ \quad (0. \, \rightarrow \, 0.) $ 

Nombre de Schmidt : $\text{Sc} =  $
\begin{numerical}[points=1] 
\item[tolerance={0.0000000000e+00}] 0.0000000000e+00 
\end{numerical} 
 $\,$ 
 $ \quad (0.0 \, \rightarrow \, 0.0) $ 

Nombre de Sherwood : $\text{Sh} =  $
\begin{numerical}[points=2] 
\item[tolerance={0.0000000000e+00}] 0.0000000000e+00 
\end{numerical} 
 $\,$ 
 $ \quad (0. \, \rightarrow \, 0.) $ 

Coefficient d'échange : $k_m =  $
\begin{numerical}[points=1] 
\item[tolerance={0.0000000000e+00}] 0.0000000000e+00 
\end{numerical} 
 $\,  \mathrm{m}\,  \mathrm{s}^{-1}$ 
 $ \quad (0.0 \, \rightarrow \, 0.0) $ 

Concentration en vapeur à la surface : $C_{V, \text{surface}} =  $
\begin{numerical}[points=2] 
\item[tolerance={0.0000000000e+00}] 0.0000000000e+00 
\end{numerical} 
 $\,  \mathrm{mol}\,  \mathrm{m}^{-3}$ 
 $ \quad (0.00 \, \rightarrow \, 0.00) $ 

Concentration en vapeur incidente : $C_{V, \infty} =  $
\begin{numerical}[points=2] 
\item[tolerance={0.0000000000e+00}] 0.0000000000e+00 
\end{numerical} 
 $\,  \mathrm{mol}\,  \mathrm{m}^{-3}$ 
 $ \quad (0.00 \, \rightarrow \, 0.00) $ 

Flux massique d'évaporation : $\dot{m}_V =  $
\begin{numerical}[points=1] 
\item[tolerance={0.0000000000e+00}] 0.0000000000e+00 
\end{numerical} 
 $\,  \mathrm{kg}\,  \mathrm{s}^{-1}$ 
 $ \quad (0.0 \, \rightarrow \, 0.0) $ 

 

On donne l'épaisseur du film d'eau à la surface du fruit $e = 1.0\,  \mathrm{mm} $

Calculez le temps total d'évaporation : $\tau =  $
\begin{numerical}[points=2] 
\item[tolerance={0.0000000000e+00}] 0.0000000000e+00 
\end{numerical} 
 $\,  \mathrm{s}$ 
 $ \quad (0. \, \rightarrow \, 0.) $ 

\end{cloze} 


 \begin{cloze}{Séchage d un fruit} 
Un fruit sphérique de diamètre $D$, humide en surface, est séché par de l'air à la température $T_\infty$, d'humidité relative $\psi$, s'écoulant à la vitesse $U_\infty$.

On veut calculer la vitesse d'évaporation de la pellicule d'eau en surface.

 

Les données sont les suivantes :

 

Diamètre $D = 9.0\,  \mathrm{cm} $

Température $T = 49.0\,  \mathrm{^\circ\mathrm{C}} $

Humidité relative $\psi = 49.\, \% $

Vitesse $U_\infty = 0.0\,  \mathrm{m}\,  \mathrm{s}^{-1} $

Pression atmosphérique $p_{\text{atm}} = 101300.\,  \mathrm{Pa} $

Pression de valeur saturante de l’eau à la température donnée $p_{\text{sat}}(T) = 0.\,  \mathrm{Pa} $

Coefficient de diffusion air/vapeur d’eau $D_{AV} = 0.0\,  \mathrm{m}^{2}\,  \mathrm{s}^{-1} $

Viscosité cinématique de l’air $\nu = 0.0\,  \mathrm{m}^{2}\,  \mathrm{s}^{-1} $

Ces données vous sont personnelles.

 

Calculez les grandeurs ci-dessous. La corrélation utilisée doit \textbf{faire intervenir le nombre de Schmidt}.

Dans vos réponses, utilisez les notations scientifiques si besoin est ($6.34\, 10^{-5}$ s'écrit 6.34e-5 et $10^{3}$ s'écrit 1e3).

Vous avez droit à une marge d'erreur relative de $5.0\, \% $

Des fourchettes indicatives vous sont fournies en face de chaque réponse. Ce ne sont que des ordres de grandeur pour vous inciter à vérifier vos calculs avant de valider vos réponses, et la bonne réponse peut sortir légèrement de l'intervalle indiqué.

 

Nombre de Reynolds : $\text{Re} =  $
\begin{numerical}[points=1] 
\item[tolerance={0.0000000000e+00}] 0.0000000000e+00 
\end{numerical} 
 $\,$ 
 $ \quad (0. \, \rightarrow \, 0.) $ 

Nombre de Schmidt : $\text{Sc} =  $
\begin{numerical}[points=1] 
\item[tolerance={0.0000000000e+00}] 0.0000000000e+00 
\end{numerical} 
 $\,$ 
 $ \quad (0.0 \, \rightarrow \, 0.0) $ 

Nombre de Sherwood : $\text{Sh} =  $
\begin{numerical}[points=2] 
\item[tolerance={0.0000000000e+00}] 0.0000000000e+00 
\end{numerical} 
 $\,$ 
 $ \quad (0. \, \rightarrow \, 0.) $ 

Coefficient d'échange : $k_m =  $
\begin{numerical}[points=1] 
\item[tolerance={0.0000000000e+00}] 0.0000000000e+00 
\end{numerical} 
 $\,  \mathrm{m}\,  \mathrm{s}^{-1}$ 
 $ \quad (0.0 \, \rightarrow \, 0.0) $ 

Concentration en vapeur à la surface : $C_{V, \text{surface}} =  $
\begin{numerical}[points=2] 
\item[tolerance={0.0000000000e+00}] 0.0000000000e+00 
\end{numerical} 
 $\,  \mathrm{mol}\,  \mathrm{m}^{-3}$ 
 $ \quad (0.00 \, \rightarrow \, 0.00) $ 

Concentration en vapeur incidente : $C_{V, \infty} =  $
\begin{numerical}[points=2] 
\item[tolerance={0.0000000000e+00}] 0.0000000000e+00 
\end{numerical} 
 $\,  \mathrm{mol}\,  \mathrm{m}^{-3}$ 
 $ \quad (0.00 \, \rightarrow \, 0.00) $ 

Flux massique d'évaporation : $\dot{m}_V =  $
\begin{numerical}[points=1] 
\item[tolerance={0.0000000000e+00}] 0.0000000000e+00 
\end{numerical} 
 $\,  \mathrm{kg}\,  \mathrm{s}^{-1}$ 
 $ \quad (0.0 \, \rightarrow \, 0.0) $ 

 

On donne l'épaisseur du film d'eau à la surface du fruit $e = 1.0\,  \mathrm{mm} $

Calculez le temps total d'évaporation : $\tau =  $
\begin{numerical}[points=2] 
\item[tolerance={0.0000000000e+00}] 0.0000000000e+00 
\end{numerical} 
 $\,  \mathrm{s}$ 
 $ \quad (0. \, \rightarrow \, 0.) $ 

\end{cloze} 


 \begin{cloze}{Séchage d un fruit} 
Un fruit sphérique de diamètre $D$, humide en surface, est séché par de l'air à la température $T_\infty$, d'humidité relative $\psi$, s'écoulant à la vitesse $U_\infty$.

On veut calculer la vitesse d'évaporation de la pellicule d'eau en surface.

 

Les données sont les suivantes :

 

Diamètre $D = 7.0\,  \mathrm{cm} $

Température $T = 40.0\,  \mathrm{^\circ\mathrm{C}} $

Humidité relative $\psi = 34.\, \% $

Vitesse $U_\infty = 0.0\,  \mathrm{m}\,  \mathrm{s}^{-1} $

Pression atmosphérique $p_{\text{atm}} = 101300.\,  \mathrm{Pa} $

Pression de valeur saturante de l’eau à la température donnée $p_{\text{sat}}(T) = 0.\,  \mathrm{Pa} $

Coefficient de diffusion air/vapeur d’eau $D_{AV} = 0.0\,  \mathrm{m}^{2}\,  \mathrm{s}^{-1} $

Viscosité cinématique de l’air $\nu = 0.0\,  \mathrm{m}^{2}\,  \mathrm{s}^{-1} $

Ces données vous sont personnelles.

 

Calculez les grandeurs ci-dessous. La corrélation utilisée doit \textbf{faire intervenir le nombre de Schmidt}.

Dans vos réponses, utilisez les notations scientifiques si besoin est ($6.34\, 10^{-5}$ s'écrit 6.34e-5 et $10^{3}$ s'écrit 1e3).

Vous avez droit à une marge d'erreur relative de $5.0\, \% $

Des fourchettes indicatives vous sont fournies en face de chaque réponse. Ce ne sont que des ordres de grandeur pour vous inciter à vérifier vos calculs avant de valider vos réponses, et la bonne réponse peut sortir légèrement de l'intervalle indiqué.

 

Nombre de Reynolds : $\text{Re} =  $
\begin{numerical}[points=1] 
\item[tolerance={0.0000000000e+00}] 0.0000000000e+00 
\end{numerical} 
 $\,$ 
 $ \quad (0. \, \rightarrow \, 0.) $ 

Nombre de Schmidt : $\text{Sc} =  $
\begin{numerical}[points=1] 
\item[tolerance={0.0000000000e+00}] 0.0000000000e+00 
\end{numerical} 
 $\,$ 
 $ \quad (0.0 \, \rightarrow \, 0.0) $ 

Nombre de Sherwood : $\text{Sh} =  $
\begin{numerical}[points=2] 
\item[tolerance={0.0000000000e+00}] 0.0000000000e+00 
\end{numerical} 
 $\,$ 
 $ \quad (0. \, \rightarrow \, 0.) $ 

Coefficient d'échange : $k_m =  $
\begin{numerical}[points=1] 
\item[tolerance={0.0000000000e+00}] 0.0000000000e+00 
\end{numerical} 
 $\,  \mathrm{m}\,  \mathrm{s}^{-1}$ 
 $ \quad (0.0 \, \rightarrow \, 0.0) $ 

Concentration en vapeur à la surface : $C_{V, \text{surface}} =  $
\begin{numerical}[points=2] 
\item[tolerance={0.0000000000e+00}] 0.0000000000e+00 
\end{numerical} 
 $\,  \mathrm{mol}\,  \mathrm{m}^{-3}$ 
 $ \quad (0.00 \, \rightarrow \, 0.00) $ 

Concentration en vapeur incidente : $C_{V, \infty} =  $
\begin{numerical}[points=2] 
\item[tolerance={0.0000000000e+00}] 0.0000000000e+00 
\end{numerical} 
 $\,  \mathrm{mol}\,  \mathrm{m}^{-3}$ 
 $ \quad (0.00 \, \rightarrow \, 0.00) $ 

Flux massique d'évaporation : $\dot{m}_V =  $
\begin{numerical}[points=1] 
\item[tolerance={0.0000000000e+00}] 0.0000000000e+00 
\end{numerical} 
 $\,  \mathrm{kg}\,  \mathrm{s}^{-1}$ 
 $ \quad (0.0 \, \rightarrow \, 0.0) $ 

 

On donne l'épaisseur du film d'eau à la surface du fruit $e = 1.0\,  \mathrm{mm} $

Calculez le temps total d'évaporation : $\tau =  $
\begin{numerical}[points=2] 
\item[tolerance={0.0000000000e+00}] 0.0000000000e+00 
\end{numerical} 
 $\,  \mathrm{s}$ 
 $ \quad (0. \, \rightarrow \, 0.) $ 

\end{cloze} 


 \begin{cloze}{Séchage d un fruit} 
Un fruit sphérique de diamètre $D$, humide en surface, est séché par de l'air à la température $T_\infty$, d'humidité relative $\psi$, s'écoulant à la vitesse $U_\infty$.

On veut calculer la vitesse d'évaporation de la pellicule d'eau en surface.

 

Les données sont les suivantes :

 

Diamètre $D = 9.0\,  \mathrm{cm} $

Température $T = 40.0\,  \mathrm{^\circ\mathrm{C}} $

Humidité relative $\psi = 42.\, \% $

Vitesse $U_\infty = 0.0\,  \mathrm{m}\,  \mathrm{s}^{-1} $

Pression atmosphérique $p_{\text{atm}} = 101300.\,  \mathrm{Pa} $

Pression de valeur saturante de l’eau à la température donnée $p_{\text{sat}}(T) = 0.\,  \mathrm{Pa} $

Coefficient de diffusion air/vapeur d’eau $D_{AV} = 0.0\,  \mathrm{m}^{2}\,  \mathrm{s}^{-1} $

Viscosité cinématique de l’air $\nu = 0.0\,  \mathrm{m}^{2}\,  \mathrm{s}^{-1} $

Ces données vous sont personnelles.

 

Calculez les grandeurs ci-dessous. La corrélation utilisée doit \textbf{faire intervenir le nombre de Schmidt}.

Dans vos réponses, utilisez les notations scientifiques si besoin est ($6.34\, 10^{-5}$ s'écrit 6.34e-5 et $10^{3}$ s'écrit 1e3).

Vous avez droit à une marge d'erreur relative de $5.0\, \% $

Des fourchettes indicatives vous sont fournies en face de chaque réponse. Ce ne sont que des ordres de grandeur pour vous inciter à vérifier vos calculs avant de valider vos réponses, et la bonne réponse peut sortir légèrement de l'intervalle indiqué.

 

Nombre de Reynolds : $\text{Re} =  $
\begin{numerical}[points=1] 
\item[tolerance={0.0000000000e+00}] 0.0000000000e+00 
\end{numerical} 
 $\,$ 
 $ \quad (0. \, \rightarrow \, 0.) $ 

Nombre de Schmidt : $\text{Sc} =  $
\begin{numerical}[points=1] 
\item[tolerance={0.0000000000e+00}] 0.0000000000e+00 
\end{numerical} 
 $\,$ 
 $ \quad (0.0 \, \rightarrow \, 0.0) $ 

Nombre de Sherwood : $\text{Sh} =  $
\begin{numerical}[points=2] 
\item[tolerance={0.0000000000e+00}] 0.0000000000e+00 
\end{numerical} 
 $\,$ 
 $ \quad (0. \, \rightarrow \, 0.) $ 

Coefficient d'échange : $k_m =  $
\begin{numerical}[points=1] 
\item[tolerance={0.0000000000e+00}] 0.0000000000e+00 
\end{numerical} 
 $\,  \mathrm{m}\,  \mathrm{s}^{-1}$ 
 $ \quad (0.0 \, \rightarrow \, 0.0) $ 

Concentration en vapeur à la surface : $C_{V, \text{surface}} =  $
\begin{numerical}[points=2] 
\item[tolerance={0.0000000000e+00}] 0.0000000000e+00 
\end{numerical} 
 $\,  \mathrm{mol}\,  \mathrm{m}^{-3}$ 
 $ \quad (0.00 \, \rightarrow \, 0.00) $ 

Concentration en vapeur incidente : $C_{V, \infty} =  $
\begin{numerical}[points=2] 
\item[tolerance={0.0000000000e+00}] 0.0000000000e+00 
\end{numerical} 
 $\,  \mathrm{mol}\,  \mathrm{m}^{-3}$ 
 $ \quad (0.00 \, \rightarrow \, 0.00) $ 

Flux massique d'évaporation : $\dot{m}_V =  $
\begin{numerical}[points=1] 
\item[tolerance={0.0000000000e+00}] 0.0000000000e+00 
\end{numerical} 
 $\,  \mathrm{kg}\,  \mathrm{s}^{-1}$ 
 $ \quad (0.0 \, \rightarrow \, 0.0) $ 

 

On donne l'épaisseur du film d'eau à la surface du fruit $e = 1.0\,  \mathrm{mm} $

Calculez le temps total d'évaporation : $\tau =  $
\begin{numerical}[points=2] 
\item[tolerance={0.0000000000e+00}] 0.0000000000e+00 
\end{numerical} 
 $\,  \mathrm{s}$ 
 $ \quad (0. \, \rightarrow \, 0.) $ 

\end{cloze} 


 \begin{cloze}{Séchage d un fruit} 
Un fruit sphérique de diamètre $D$, humide en surface, est séché par de l'air à la température $T_\infty$, d'humidité relative $\psi$, s'écoulant à la vitesse $U_\infty$.

On veut calculer la vitesse d'évaporation de la pellicule d'eau en surface.

 

Les données sont les suivantes :

 

Diamètre $D = 6.0\,  \mathrm{cm} $

Température $T = 47.0\,  \mathrm{^\circ\mathrm{C}} $

Humidité relative $\psi = 45.\, \% $

Vitesse $U_\infty = 0.0\,  \mathrm{m}\,  \mathrm{s}^{-1} $

Pression atmosphérique $p_{\text{atm}} = 101300.\,  \mathrm{Pa} $

Pression de valeur saturante de l’eau à la température donnée $p_{\text{sat}}(T) = 0.\,  \mathrm{Pa} $

Coefficient de diffusion air/vapeur d’eau $D_{AV} = 0.0\,  \mathrm{m}^{2}\,  \mathrm{s}^{-1} $

Viscosité cinématique de l’air $\nu = 0.0\,  \mathrm{m}^{2}\,  \mathrm{s}^{-1} $

Ces données vous sont personnelles.

 

Calculez les grandeurs ci-dessous. La corrélation utilisée doit \textbf{faire intervenir le nombre de Schmidt}.

Dans vos réponses, utilisez les notations scientifiques si besoin est ($6.34\, 10^{-5}$ s'écrit 6.34e-5 et $10^{3}$ s'écrit 1e3).

Vous avez droit à une marge d'erreur relative de $5.0\, \% $

Des fourchettes indicatives vous sont fournies en face de chaque réponse. Ce ne sont que des ordres de grandeur pour vous inciter à vérifier vos calculs avant de valider vos réponses, et la bonne réponse peut sortir légèrement de l'intervalle indiqué.

 

Nombre de Reynolds : $\text{Re} =  $
\begin{numerical}[points=1] 
\item[tolerance={0.0000000000e+00}] 0.0000000000e+00 
\end{numerical} 
 $\,$ 
 $ \quad (0. \, \rightarrow \, 0.) $ 

Nombre de Schmidt : $\text{Sc} =  $
\begin{numerical}[points=1] 
\item[tolerance={0.0000000000e+00}] 0.0000000000e+00 
\end{numerical} 
 $\,$ 
 $ \quad (0.0 \, \rightarrow \, 0.0) $ 

Nombre de Sherwood : $\text{Sh} =  $
\begin{numerical}[points=2] 
\item[tolerance={0.0000000000e+00}] 0.0000000000e+00 
\end{numerical} 
 $\,$ 
 $ \quad (0. \, \rightarrow \, 0.) $ 

Coefficient d'échange : $k_m =  $
\begin{numerical}[points=1] 
\item[tolerance={0.0000000000e+00}] 0.0000000000e+00 
\end{numerical} 
 $\,  \mathrm{m}\,  \mathrm{s}^{-1}$ 
 $ \quad (0.0 \, \rightarrow \, 0.0) $ 

Concentration en vapeur à la surface : $C_{V, \text{surface}} =  $
\begin{numerical}[points=2] 
\item[tolerance={0.0000000000e+00}] 0.0000000000e+00 
\end{numerical} 
 $\,  \mathrm{mol}\,  \mathrm{m}^{-3}$ 
 $ \quad (0.00 \, \rightarrow \, 0.00) $ 

Concentration en vapeur incidente : $C_{V, \infty} =  $
\begin{numerical}[points=2] 
\item[tolerance={0.0000000000e+00}] 0.0000000000e+00 
\end{numerical} 
 $\,  \mathrm{mol}\,  \mathrm{m}^{-3}$ 
 $ \quad (0.00 \, \rightarrow \, 0.00) $ 

Flux massique d'évaporation : $\dot{m}_V =  $
\begin{numerical}[points=1] 
\item[tolerance={0.0000000000e+00}] 0.0000000000e+00 
\end{numerical} 
 $\,  \mathrm{kg}\,  \mathrm{s}^{-1}$ 
 $ \quad (0.0 \, \rightarrow \, 0.0) $ 

 

On donne l'épaisseur du film d'eau à la surface du fruit $e = 1.0\,  \mathrm{mm} $

Calculez le temps total d'évaporation : $\tau =  $
\begin{numerical}[points=2] 
\item[tolerance={0.0000000000e+00}] 0.0000000000e+00 
\end{numerical} 
 $\,  \mathrm{s}$ 
 $ \quad (0. \, \rightarrow \, 0.) $ 

\end{cloze} 


 \begin{cloze}{Séchage d un fruit} 
Un fruit sphérique de diamètre $D$, humide en surface, est séché par de l'air à la température $T_\infty$, d'humidité relative $\psi$, s'écoulant à la vitesse $U_\infty$.

On veut calculer la vitesse d'évaporation de la pellicule d'eau en surface.

 

Les données sont les suivantes :

 

Diamètre $D = 7.0\,  \mathrm{cm} $

Température $T = 39.0\,  \mathrm{^\circ\mathrm{C}} $

Humidité relative $\psi = 41.\, \% $

Vitesse $U_\infty = 0.0\,  \mathrm{m}\,  \mathrm{s}^{-1} $

Pression atmosphérique $p_{\text{atm}} = 101300.\,  \mathrm{Pa} $

Pression de valeur saturante de l’eau à la température donnée $p_{\text{sat}}(T) = 0.\,  \mathrm{Pa} $

Coefficient de diffusion air/vapeur d’eau $D_{AV} = 0.0\,  \mathrm{m}^{2}\,  \mathrm{s}^{-1} $

Viscosité cinématique de l’air $\nu = 0.0\,  \mathrm{m}^{2}\,  \mathrm{s}^{-1} $

Ces données vous sont personnelles.

 

Calculez les grandeurs ci-dessous. La corrélation utilisée doit \textbf{faire intervenir le nombre de Schmidt}.

Dans vos réponses, utilisez les notations scientifiques si besoin est ($6.34\, 10^{-5}$ s'écrit 6.34e-5 et $10^{3}$ s'écrit 1e3).

Vous avez droit à une marge d'erreur relative de $5.0\, \% $

Des fourchettes indicatives vous sont fournies en face de chaque réponse. Ce ne sont que des ordres de grandeur pour vous inciter à vérifier vos calculs avant de valider vos réponses, et la bonne réponse peut sortir légèrement de l'intervalle indiqué.

 

Nombre de Reynolds : $\text{Re} =  $
\begin{numerical}[points=1] 
\item[tolerance={0.0000000000e+00}] 0.0000000000e+00 
\end{numerical} 
 $\,$ 
 $ \quad (0. \, \rightarrow \, 0.) $ 

Nombre de Schmidt : $\text{Sc} =  $
\begin{numerical}[points=1] 
\item[tolerance={0.0000000000e+00}] 0.0000000000e+00 
\end{numerical} 
 $\,$ 
 $ \quad (0.0 \, \rightarrow \, 0.0) $ 

Nombre de Sherwood : $\text{Sh} =  $
\begin{numerical}[points=2] 
\item[tolerance={0.0000000000e+00}] 0.0000000000e+00 
\end{numerical} 
 $\,$ 
 $ \quad (0. \, \rightarrow \, 0.) $ 

Coefficient d'échange : $k_m =  $
\begin{numerical}[points=1] 
\item[tolerance={0.0000000000e+00}] 0.0000000000e+00 
\end{numerical} 
 $\,  \mathrm{m}\,  \mathrm{s}^{-1}$ 
 $ \quad (0.0 \, \rightarrow \, 0.0) $ 

Concentration en vapeur à la surface : $C_{V, \text{surface}} =  $
\begin{numerical}[points=2] 
\item[tolerance={0.0000000000e+00}] 0.0000000000e+00 
\end{numerical} 
 $\,  \mathrm{mol}\,  \mathrm{m}^{-3}$ 
 $ \quad (0.00 \, \rightarrow \, 0.00) $ 

Concentration en vapeur incidente : $C_{V, \infty} =  $
\begin{numerical}[points=2] 
\item[tolerance={0.0000000000e+00}] 0.0000000000e+00 
\end{numerical} 
 $\,  \mathrm{mol}\,  \mathrm{m}^{-3}$ 
 $ \quad (0.00 \, \rightarrow \, 0.00) $ 

Flux massique d'évaporation : $\dot{m}_V =  $
\begin{numerical}[points=1] 
\item[tolerance={0.0000000000e+00}] 0.0000000000e+00 
\end{numerical} 
 $\,  \mathrm{kg}\,  \mathrm{s}^{-1}$ 
 $ \quad (0.0 \, \rightarrow \, 0.0) $ 

 

On donne l'épaisseur du film d'eau à la surface du fruit $e = 1.0\,  \mathrm{mm} $

Calculez le temps total d'évaporation : $\tau =  $
\begin{numerical}[points=2] 
\item[tolerance={0.0000000000e+00}] 0.0000000000e+00 
\end{numerical} 
 $\,  \mathrm{s}$ 
 $ \quad (0. \, \rightarrow \, 0.) $ 

\end{cloze} 


 \begin{cloze}{Séchage d un fruit} 
Un fruit sphérique de diamètre $D$, humide en surface, est séché par de l'air à la température $T_\infty$, d'humidité relative $\psi$, s'écoulant à la vitesse $U_\infty$.

On veut calculer la vitesse d'évaporation de la pellicule d'eau en surface.

 

Les données sont les suivantes :

 

Diamètre $D = 4.0\,  \mathrm{cm} $

Température $T = 33.0\,  \mathrm{^\circ\mathrm{C}} $

Humidité relative $\psi = 44.\, \% $

Vitesse $U_\infty = 0.0\,  \mathrm{m}\,  \mathrm{s}^{-1} $

Pression atmosphérique $p_{\text{atm}} = 101300.\,  \mathrm{Pa} $

Pression de valeur saturante de l’eau à la température donnée $p_{\text{sat}}(T) = 0.\,  \mathrm{Pa} $

Coefficient de diffusion air/vapeur d’eau $D_{AV} = 0.0\,  \mathrm{m}^{2}\,  \mathrm{s}^{-1} $

Viscosité cinématique de l’air $\nu = 0.0\,  \mathrm{m}^{2}\,  \mathrm{s}^{-1} $

Ces données vous sont personnelles.

 

Calculez les grandeurs ci-dessous. La corrélation utilisée doit \textbf{faire intervenir le nombre de Schmidt}.

Dans vos réponses, utilisez les notations scientifiques si besoin est ($6.34\, 10^{-5}$ s'écrit 6.34e-5 et $10^{3}$ s'écrit 1e3).

Vous avez droit à une marge d'erreur relative de $5.0\, \% $

Des fourchettes indicatives vous sont fournies en face de chaque réponse. Ce ne sont que des ordres de grandeur pour vous inciter à vérifier vos calculs avant de valider vos réponses, et la bonne réponse peut sortir légèrement de l'intervalle indiqué.

 

Nombre de Reynolds : $\text{Re} =  $
\begin{numerical}[points=1] 
\item[tolerance={0.0000000000e+00}] 0.0000000000e+00 
\end{numerical} 
 $\,$ 
 $ \quad (0. \, \rightarrow \, 0.) $ 

Nombre de Schmidt : $\text{Sc} =  $
\begin{numerical}[points=1] 
\item[tolerance={0.0000000000e+00}] 0.0000000000e+00 
\end{numerical} 
 $\,$ 
 $ \quad (0.0 \, \rightarrow \, 0.0) $ 

Nombre de Sherwood : $\text{Sh} =  $
\begin{numerical}[points=2] 
\item[tolerance={0.0000000000e+00}] 0.0000000000e+00 
\end{numerical} 
 $\,$ 
 $ \quad (0. \, \rightarrow \, 0.) $ 

Coefficient d'échange : $k_m =  $
\begin{numerical}[points=1] 
\item[tolerance={0.0000000000e+00}] 0.0000000000e+00 
\end{numerical} 
 $\,  \mathrm{m}\,  \mathrm{s}^{-1}$ 
 $ \quad (0.0 \, \rightarrow \, 0.0) $ 

Concentration en vapeur à la surface : $C_{V, \text{surface}} =  $
\begin{numerical}[points=2] 
\item[tolerance={0.0000000000e+00}] 0.0000000000e+00 
\end{numerical} 
 $\,  \mathrm{mol}\,  \mathrm{m}^{-3}$ 
 $ \quad (0.00 \, \rightarrow \, 0.00) $ 

Concentration en vapeur incidente : $C_{V, \infty} =  $
\begin{numerical}[points=2] 
\item[tolerance={0.0000000000e+00}] 0.0000000000e+00 
\end{numerical} 
 $\,  \mathrm{mol}\,  \mathrm{m}^{-3}$ 
 $ \quad (0.00 \, \rightarrow \, 0.00) $ 

Flux massique d'évaporation : $\dot{m}_V =  $
\begin{numerical}[points=1] 
\item[tolerance={0.0000000000e+00}] 0.0000000000e+00 
\end{numerical} 
 $\,  \mathrm{kg}\,  \mathrm{s}^{-1}$ 
 $ \quad (0.0 \, \rightarrow \, 0.0) $ 

 

On donne l'épaisseur du film d'eau à la surface du fruit $e = 1.0\,  \mathrm{mm} $

Calculez le temps total d'évaporation : $\tau =  $
\begin{numerical}[points=2] 
\item[tolerance={0.0000000000e+00}] 0.0000000000e+00 
\end{numerical} 
 $\,  \mathrm{s}$ 
 $ \quad (0. \, \rightarrow \, 0.) $ 

\end{cloze} 


 \begin{cloze}{Séchage d un fruit} 
Un fruit sphérique de diamètre $D$, humide en surface, est séché par de l'air à la température $T_\infty$, d'humidité relative $\psi$, s'écoulant à la vitesse $U_\infty$.

On veut calculer la vitesse d'évaporation de la pellicule d'eau en surface.

 

Les données sont les suivantes :

 

Diamètre $D = 9.0\,  \mathrm{cm} $

Température $T = 48.0\,  \mathrm{^\circ\mathrm{C}} $

Humidité relative $\psi = 40.\, \% $

Vitesse $U_\infty = 0.0\,  \mathrm{m}\,  \mathrm{s}^{-1} $

Pression atmosphérique $p_{\text{atm}} = 101300.\,  \mathrm{Pa} $

Pression de valeur saturante de l’eau à la température donnée $p_{\text{sat}}(T) = 0.\,  \mathrm{Pa} $

Coefficient de diffusion air/vapeur d’eau $D_{AV} = 0.0\,  \mathrm{m}^{2}\,  \mathrm{s}^{-1} $

Viscosité cinématique de l’air $\nu = 0.0\,  \mathrm{m}^{2}\,  \mathrm{s}^{-1} $

Ces données vous sont personnelles.

 

Calculez les grandeurs ci-dessous. La corrélation utilisée doit \textbf{faire intervenir le nombre de Schmidt}.

Dans vos réponses, utilisez les notations scientifiques si besoin est ($6.34\, 10^{-5}$ s'écrit 6.34e-5 et $10^{3}$ s'écrit 1e3).

Vous avez droit à une marge d'erreur relative de $5.0\, \% $

Des fourchettes indicatives vous sont fournies en face de chaque réponse. Ce ne sont que des ordres de grandeur pour vous inciter à vérifier vos calculs avant de valider vos réponses, et la bonne réponse peut sortir légèrement de l'intervalle indiqué.

 

Nombre de Reynolds : $\text{Re} =  $
\begin{numerical}[points=1] 
\item[tolerance={0.0000000000e+00}] 0.0000000000e+00 
\end{numerical} 
 $\,$ 
 $ \quad (0. \, \rightarrow \, 0.) $ 

Nombre de Schmidt : $\text{Sc} =  $
\begin{numerical}[points=1] 
\item[tolerance={0.0000000000e+00}] 0.0000000000e+00 
\end{numerical} 
 $\,$ 
 $ \quad (0.0 \, \rightarrow \, 0.0) $ 

Nombre de Sherwood : $\text{Sh} =  $
\begin{numerical}[points=2] 
\item[tolerance={0.0000000000e+00}] 0.0000000000e+00 
\end{numerical} 
 $\,$ 
 $ \quad (0. \, \rightarrow \, 0.) $ 

Coefficient d'échange : $k_m =  $
\begin{numerical}[points=1] 
\item[tolerance={0.0000000000e+00}] 0.0000000000e+00 
\end{numerical} 
 $\,  \mathrm{m}\,  \mathrm{s}^{-1}$ 
 $ \quad (0.0 \, \rightarrow \, 0.0) $ 

Concentration en vapeur à la surface : $C_{V, \text{surface}} =  $
\begin{numerical}[points=2] 
\item[tolerance={0.0000000000e+00}] 0.0000000000e+00 
\end{numerical} 
 $\,  \mathrm{mol}\,  \mathrm{m}^{-3}$ 
 $ \quad (0.00 \, \rightarrow \, 0.00) $ 

Concentration en vapeur incidente : $C_{V, \infty} =  $
\begin{numerical}[points=2] 
\item[tolerance={0.0000000000e+00}] 0.0000000000e+00 
\end{numerical} 
 $\,  \mathrm{mol}\,  \mathrm{m}^{-3}$ 
 $ \quad (0.00 \, \rightarrow \, 0.00) $ 

Flux massique d'évaporation : $\dot{m}_V =  $
\begin{numerical}[points=1] 
\item[tolerance={0.0000000000e+00}] 0.0000000000e+00 
\end{numerical} 
 $\,  \mathrm{kg}\,  \mathrm{s}^{-1}$ 
 $ \quad (0.0 \, \rightarrow \, 0.0) $ 

 

On donne l'épaisseur du film d'eau à la surface du fruit $e = 1.0\,  \mathrm{mm} $

Calculez le temps total d'évaporation : $\tau =  $
\begin{numerical}[points=2] 
\item[tolerance={0.0000000000e+00}] 0.0000000000e+00 
\end{numerical} 
 $\,  \mathrm{s}$ 
 $ \quad (0. \, \rightarrow \, 0.) $ 

\end{cloze} 


 \begin{cloze}{Séchage d un fruit} 
Un fruit sphérique de diamètre $D$, humide en surface, est séché par de l'air à la température $T_\infty$, d'humidité relative $\psi$, s'écoulant à la vitesse $U_\infty$.

On veut calculer la vitesse d'évaporation de la pellicule d'eau en surface.

 

Les données sont les suivantes :

 

Diamètre $D = 7.0\,  \mathrm{cm} $

Température $T = 35.0\,  \mathrm{^\circ\mathrm{C}} $

Humidité relative $\psi = 53.\, \% $

Vitesse $U_\infty = 0.0\,  \mathrm{m}\,  \mathrm{s}^{-1} $

Pression atmosphérique $p_{\text{atm}} = 101300.\,  \mathrm{Pa} $

Pression de valeur saturante de l’eau à la température donnée $p_{\text{sat}}(T) = 0.\,  \mathrm{Pa} $

Coefficient de diffusion air/vapeur d’eau $D_{AV} = 0.0\,  \mathrm{m}^{2}\,  \mathrm{s}^{-1} $

Viscosité cinématique de l’air $\nu = 0.0\,  \mathrm{m}^{2}\,  \mathrm{s}^{-1} $

Ces données vous sont personnelles.

 

Calculez les grandeurs ci-dessous. La corrélation utilisée doit \textbf{faire intervenir le nombre de Schmidt}.

Dans vos réponses, utilisez les notations scientifiques si besoin est ($6.34\, 10^{-5}$ s'écrit 6.34e-5 et $10^{3}$ s'écrit 1e3).

Vous avez droit à une marge d'erreur relative de $5.0\, \% $

Des fourchettes indicatives vous sont fournies en face de chaque réponse. Ce ne sont que des ordres de grandeur pour vous inciter à vérifier vos calculs avant de valider vos réponses, et la bonne réponse peut sortir légèrement de l'intervalle indiqué.

 

Nombre de Reynolds : $\text{Re} =  $
\begin{numerical}[points=1] 
\item[tolerance={0.0000000000e+00}] 0.0000000000e+00 
\end{numerical} 
 $\,$ 
 $ \quad (0. \, \rightarrow \, 0.) $ 

Nombre de Schmidt : $\text{Sc} =  $
\begin{numerical}[points=1] 
\item[tolerance={0.0000000000e+00}] 0.0000000000e+00 
\end{numerical} 
 $\,$ 
 $ \quad (0.0 \, \rightarrow \, 0.0) $ 

Nombre de Sherwood : $\text{Sh} =  $
\begin{numerical}[points=2] 
\item[tolerance={0.0000000000e+00}] 0.0000000000e+00 
\end{numerical} 
 $\,$ 
 $ \quad (0. \, \rightarrow \, 0.) $ 

Coefficient d'échange : $k_m =  $
\begin{numerical}[points=1] 
\item[tolerance={0.0000000000e+00}] 0.0000000000e+00 
\end{numerical} 
 $\,  \mathrm{m}\,  \mathrm{s}^{-1}$ 
 $ \quad (0.0 \, \rightarrow \, 0.0) $ 

Concentration en vapeur à la surface : $C_{V, \text{surface}} =  $
\begin{numerical}[points=2] 
\item[tolerance={0.0000000000e+00}] 0.0000000000e+00 
\end{numerical} 
 $\,  \mathrm{mol}\,  \mathrm{m}^{-3}$ 
 $ \quad (0.00 \, \rightarrow \, 0.00) $ 

Concentration en vapeur incidente : $C_{V, \infty} =  $
\begin{numerical}[points=2] 
\item[tolerance={0.0000000000e+00}] 0.0000000000e+00 
\end{numerical} 
 $\,  \mathrm{mol}\,  \mathrm{m}^{-3}$ 
 $ \quad (0.00 \, \rightarrow \, 0.00) $ 

Flux massique d'évaporation : $\dot{m}_V =  $
\begin{numerical}[points=1] 
\item[tolerance={0.0000000000e+00}] 0.0000000000e+00 
\end{numerical} 
 $\,  \mathrm{kg}\,  \mathrm{s}^{-1}$ 
 $ \quad (0.0 \, \rightarrow \, 0.0) $ 

 

On donne l'épaisseur du film d'eau à la surface du fruit $e = 1.0\,  \mathrm{mm} $

Calculez le temps total d'évaporation : $\tau =  $
\begin{numerical}[points=2] 
\item[tolerance={0.0000000000e+00}] 0.0000000000e+00 
\end{numerical} 
 $\,  \mathrm{s}$ 
 $ \quad (0. \, \rightarrow \, 0.) $ 

\end{cloze} 


 \begin{cloze}{Séchage d un fruit} 
Un fruit sphérique de diamètre $D$, humide en surface, est séché par de l'air à la température $T_\infty$, d'humidité relative $\psi$, s'écoulant à la vitesse $U_\infty$.

On veut calculer la vitesse d'évaporation de la pellicule d'eau en surface.

 

Les données sont les suivantes :

 

Diamètre $D = 4.0\,  \mathrm{cm} $

Température $T = 40.0\,  \mathrm{^\circ\mathrm{C}} $

Humidité relative $\psi = 41.\, \% $

Vitesse $U_\infty = 0.0\,  \mathrm{m}\,  \mathrm{s}^{-1} $

Pression atmosphérique $p_{\text{atm}} = 101300.\,  \mathrm{Pa} $

Pression de valeur saturante de l’eau à la température donnée $p_{\text{sat}}(T) = 0.\,  \mathrm{Pa} $

Coefficient de diffusion air/vapeur d’eau $D_{AV} = 0.0\,  \mathrm{m}^{2}\,  \mathrm{s}^{-1} $

Viscosité cinématique de l’air $\nu = 0.0\,  \mathrm{m}^{2}\,  \mathrm{s}^{-1} $

Ces données vous sont personnelles.

 

Calculez les grandeurs ci-dessous. La corrélation utilisée doit \textbf{faire intervenir le nombre de Schmidt}.

Dans vos réponses, utilisez les notations scientifiques si besoin est ($6.34\, 10^{-5}$ s'écrit 6.34e-5 et $10^{3}$ s'écrit 1e3).

Vous avez droit à une marge d'erreur relative de $5.0\, \% $

Des fourchettes indicatives vous sont fournies en face de chaque réponse. Ce ne sont que des ordres de grandeur pour vous inciter à vérifier vos calculs avant de valider vos réponses, et la bonne réponse peut sortir légèrement de l'intervalle indiqué.

 

Nombre de Reynolds : $\text{Re} =  $
\begin{numerical}[points=1] 
\item[tolerance={0.0000000000e+00}] 0.0000000000e+00 
\end{numerical} 
 $\,$ 
 $ \quad (0. \, \rightarrow \, 0.) $ 

Nombre de Schmidt : $\text{Sc} =  $
\begin{numerical}[points=1] 
\item[tolerance={0.0000000000e+00}] 0.0000000000e+00 
\end{numerical} 
 $\,$ 
 $ \quad (0.0 \, \rightarrow \, 0.0) $ 

Nombre de Sherwood : $\text{Sh} =  $
\begin{numerical}[points=2] 
\item[tolerance={0.0000000000e+00}] 0.0000000000e+00 
\end{numerical} 
 $\,$ 
 $ \quad (0. \, \rightarrow \, 0.) $ 

Coefficient d'échange : $k_m =  $
\begin{numerical}[points=1] 
\item[tolerance={0.0000000000e+00}] 0.0000000000e+00 
\end{numerical} 
 $\,  \mathrm{m}\,  \mathrm{s}^{-1}$ 
 $ \quad (0.0 \, \rightarrow \, 0.0) $ 

Concentration en vapeur à la surface : $C_{V, \text{surface}} =  $
\begin{numerical}[points=2] 
\item[tolerance={0.0000000000e+00}] 0.0000000000e+00 
\end{numerical} 
 $\,  \mathrm{mol}\,  \mathrm{m}^{-3}$ 
 $ \quad (0.00 \, \rightarrow \, 0.00) $ 

Concentration en vapeur incidente : $C_{V, \infty} =  $
\begin{numerical}[points=2] 
\item[tolerance={0.0000000000e+00}] 0.0000000000e+00 
\end{numerical} 
 $\,  \mathrm{mol}\,  \mathrm{m}^{-3}$ 
 $ \quad (0.00 \, \rightarrow \, 0.00) $ 

Flux massique d'évaporation : $\dot{m}_V =  $
\begin{numerical}[points=1] 
\item[tolerance={0.0000000000e+00}] 0.0000000000e+00 
\end{numerical} 
 $\,  \mathrm{kg}\,  \mathrm{s}^{-1}$ 
 $ \quad (0.0 \, \rightarrow \, 0.0) $ 

 

On donne l'épaisseur du film d'eau à la surface du fruit $e = 1.0\,  \mathrm{mm} $

Calculez le temps total d'évaporation : $\tau =  $
\begin{numerical}[points=2] 
\item[tolerance={0.0000000000e+00}] 0.0000000000e+00 
\end{numerical} 
 $\,  \mathrm{s}$ 
 $ \quad (0. \, \rightarrow \, 0.) $ 

\end{cloze} 


\end{quiz}
\end{document}
