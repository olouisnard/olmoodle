\documentclass[12pt]{article}

\usepackage[a4paper,landscape]{geometry}

\usepackage{moodle}

\usepackage{amsmath}
\usepackage{amssymb}
\usepackage{amsfonts}
\usepackage{amsthm,thmtools}
\usepackage{graphicx}

\usepackage{layouts}


\usepackage[utf8]{inputenc}
\usepackage[cyr]{aeguill}
\usepackage{xspace}


\usepackage[french]{babel}

\usepackage{ifthen}
\usepackage{verbatim} % Pour mettre des parties en commentaire

\usepackage{mymaths}
\usepackage{hyperref}



\begin{document}



\begin{quiz}{NUMERIQUES/PERTES DE CHARGE} 

 \begin{cloze}{Pertes de charge dans un réseau} 
Une pompe débite un liquide avec un déibt volumique $Q$ dans un réseau constitué de N coudes et d'une tuyauterie rectiligne de diamètre $D$ et de longueur $L$. La tuyauterie est rugueuse. Le fluide sort de la tuyauterie à une hauteur $h$ au-dessus de la surface libre.

Les données sont les suivantes :

Longueur : $L = 50.\,  \mathrm{m} $

Diamètre : $D = 55.\,  \mathrm{mm} $

Débit volumique : $Q = 12900.\,  \mathrm{L}\,  \mathrm{h}^{-1} $

Rugosité : $\epsilon = 140.\, \mu \mathrm{m} $

Hauteur de liquide $h = 8.0\,  \mathrm{m} $

Masse volumique du liquide $\rho = 900.\,  \mathrm{kg}\,  \mathrm{m}^{-3} $

Viscosité du liquide $\mu =  1.2 \, 10^{-3} \,  \mathrm{Pa}\,  \mathrm{s} $

 

Calculez :

 

Le nombre de Reynolds : $\text{Re} =  $
\begin{numerical}[points=1] 
\item[tolerance={3.1107557059e+03}] 6.2215114118e+04 
\end{numerical} 
 $\,$ 
 $ \quad (47338. \, - \, 79577.) $ 

La hauteur de pertes de charge : $h_v =  $
\begin{numerical}[points=1] 
\item[tolerance={1.4262784653e-01}] 2.8525569306e+00 
\end{numerical} 
 $\,  \mathrm{m}$ 
 $ \quad (1.0 \, - \, 5.4) $ 

La puissance nécessaire de la pompe : $\dot{W}_u =  $
\begin{numerical}[points=1] 
\item[tolerance={1.7167252838e+01}] 3.4334505675e+02 
\end{numerical} 
 $\,  \mathrm{W}$ 
 $ \quad (258. \, - \, 653.) $ 

(dans vos réponses, utilisez les notations scientifiques si besoin est. Par exemple $6.34\, 10^{-5}$ s'écrit 6.34e-5 et $10^{3}$ s'écrit 1e3).

Vous avez droit à une marge d'erreur relative de $5.0\, \% $

Des fourchettes indicatives vous sont fournies en face de chaque réponse. Ce ne sont que des ordres de grandeur pour vous inciter à vérifier vos calculs avant de valider vos réponses, et la bonne réponse peut sortir légèrement de l'intervalle indiqué.

\end{cloze} 


 \begin{cloze}{Pertes de charge dans un réseau} 
Une pompe débite un liquide avec un déibt volumique $Q$ dans un réseau constitué de N coudes et d'une tuyauterie rectiligne de diamètre $D$ et de longueur $L$. La tuyauterie est rugueuse. Le fluide sort de la tuyauterie à une hauteur $h$ au-dessus de la surface libre.

Les données sont les suivantes :

Longueur : $L = 60.\,  \mathrm{m} $

Diamètre : $D = 61.\,  \mathrm{mm} $

Débit volumique : $Q = 14300.\,  \mathrm{L}\,  \mathrm{h}^{-1} $

Rugosité : $\epsilon = 240.\, \mu \mathrm{m} $

Hauteur de liquide $h = 8.0\,  \mathrm{m} $

Masse volumique du liquide $\rho = 900.\,  \mathrm{kg}\,  \mathrm{m}^{-3} $

Viscosité du liquide $\mu =  1.2 \, 10^{-3} \,  \mathrm{Pa}\,  \mathrm{s} $

 

Calculez :

 

Le nombre de Reynolds : $\text{Re} =  $
\begin{numerical}[points=1] 
\item[tolerance={3.1091744347e+03}] 6.2183488694e+04 
\end{numerical} 
 $\,$ 
 $ \quad (47338. \, - \, 79577.) $ 

La hauteur de pertes de charge : $h_v =  $
\begin{numerical}[points=1] 
\item[tolerance={1.3841081275e-01}] 2.7682162551e+00 
\end{numerical} 
 $\,  \mathrm{m}$ 
 $ \quad (1.0 \, - \, 5.4) $ 

La puissance nécessaire de la pompe : $\dot{W}_u =  $
\begin{numerical}[points=1] 
\item[tolerance={1.8882471011e+01}] 3.7764942023e+02 
\end{numerical} 
 $\,  \mathrm{W}$ 
 $ \quad (258. \, - \, 653.) $ 

(dans vos réponses, utilisez les notations scientifiques si besoin est. Par exemple $6.34\, 10^{-5}$ s'écrit 6.34e-5 et $10^{3}$ s'écrit 1e3).

Vous avez droit à une marge d'erreur relative de $5.0\, \% $

Des fourchettes indicatives vous sont fournies en face de chaque réponse. Ce ne sont que des ordres de grandeur pour vous inciter à vérifier vos calculs avant de valider vos réponses, et la bonne réponse peut sortir légèrement de l'intervalle indiqué.

\end{cloze} 


 \begin{cloze}{Pertes de charge dans un réseau} 
Une pompe débite un liquide avec un déibt volumique $Q$ dans un réseau constitué de N coudes et d'une tuyauterie rectiligne de diamètre $D$ et de longueur $L$. La tuyauterie est rugueuse. Le fluide sort de la tuyauterie à une hauteur $h$ au-dessus de la surface libre.

Les données sont les suivantes :

Longueur : $L = 90.\,  \mathrm{m} $

Diamètre : $D = 56.\,  \mathrm{mm} $

Débit volumique : $Q = 10100.\,  \mathrm{L}\,  \mathrm{h}^{-1} $

Rugosité : $\epsilon = 170.\, \mu \mathrm{m} $

Hauteur de liquide $h = 8.0\,  \mathrm{m} $

Masse volumique du liquide $\rho = 900.\,  \mathrm{kg}\,  \mathrm{m}^{-3} $

Viscosité du liquide $\mu =  1.2 \, 10^{-3} \,  \mathrm{Pa}\,  \mathrm{s} $

 

Calculez :

 

Le nombre de Reynolds : $\text{Re} =  $
\begin{numerical}[points=1] 
\item[tolerance={2.3920609006e+03}] 4.7841218013e+04 
\end{numerical} 
 $\,$ 
 $ \quad (47338. \, - \, 79577.) $ 

La hauteur de pertes de charge : $h_v =  $
\begin{numerical}[points=1] 
\item[tolerance={1.5206169399e-01}] 3.0412338799e+00 
\end{numerical} 
 $\,  \mathrm{m}$ 
 $ \quad (1.0 \, - \, 5.4) $ 

La puissance nécessaire de la pompe : $\dot{W}_u =  $
\begin{numerical}[points=1] 
\item[tolerance={1.3674706176e+01}] 2.7349412351e+02 
\end{numerical} 
 $\,  \mathrm{W}$ 
 $ \quad (258. \, - \, 653.) $ 

(dans vos réponses, utilisez les notations scientifiques si besoin est. Par exemple $6.34\, 10^{-5}$ s'écrit 6.34e-5 et $10^{3}$ s'écrit 1e3).

Vous avez droit à une marge d'erreur relative de $5.0\, \% $

Des fourchettes indicatives vous sont fournies en face de chaque réponse. Ce ne sont que des ordres de grandeur pour vous inciter à vérifier vos calculs avant de valider vos réponses, et la bonne réponse peut sortir légèrement de l'intervalle indiqué.

\end{cloze} 


 \begin{cloze}{Pertes de charge dans un réseau} 
Une pompe débite un liquide avec un déibt volumique $Q$ dans un réseau constitué de N coudes et d'une tuyauterie rectiligne de diamètre $D$ et de longueur $L$. La tuyauterie est rugueuse. Le fluide sort de la tuyauterie à une hauteur $h$ au-dessus de la surface libre.

Les données sont les suivantes :

Longueur : $L = 90.\,  \mathrm{m} $

Diamètre : $D = 66.\,  \mathrm{mm} $

Débit volumique : $Q = 19800.\,  \mathrm{L}\,  \mathrm{h}^{-1} $

Rugosité : $\epsilon = 290.\, \mu \mathrm{m} $

Hauteur de liquide $h = 8.0\,  \mathrm{m} $

Masse volumique du liquide $\rho = 900.\,  \mathrm{kg}\,  \mathrm{m}^{-3} $

Viscosité du liquide $\mu =  1.2 \, 10^{-3} \,  \mathrm{Pa}\,  \mathrm{s} $

 

Calculez :

 

Le nombre de Reynolds : $\text{Re} =  $
\begin{numerical}[points=1] 
\item[tolerance={3.9788735773e+03}] 7.9577471546e+04 
\end{numerical} 
 $\,$ 
 $ \quad (47338. \, - \, 79577.) $ 

La hauteur de pertes de charge : $h_v =  $
\begin{numerical}[points=1] 
\item[tolerance={2.7321855032e-01}] 5.4643710064e+00 
\end{numerical} 
 $\,  \mathrm{m}$ 
 $ \quad (1.0 \, - \, 5.4) $ 

La puissance nécessaire de la pompe : $\dot{W}_u =  $
\begin{numerical}[points=1] 
\item[tolerance={3.2691156194e+01}] 6.5382312388e+02 
\end{numerical} 
 $\,  \mathrm{W}$ 
 $ \quad (258. \, - \, 653.) $ 

(dans vos réponses, utilisez les notations scientifiques si besoin est. Par exemple $6.34\, 10^{-5}$ s'écrit 6.34e-5 et $10^{3}$ s'écrit 1e3).

Vous avez droit à une marge d'erreur relative de $5.0\, \% $

Des fourchettes indicatives vous sont fournies en face de chaque réponse. Ce ne sont que des ordres de grandeur pour vous inciter à vérifier vos calculs avant de valider vos réponses, et la bonne réponse peut sortir légèrement de l'intervalle indiqué.

\end{cloze} 


 \begin{cloze}{Pertes de charge dans un réseau} 
Une pompe débite un liquide avec un déibt volumique $Q$ dans un réseau constitué de N coudes et d'une tuyauterie rectiligne de diamètre $D$ et de longueur $L$. La tuyauterie est rugueuse. Le fluide sort de la tuyauterie à une hauteur $h$ au-dessus de la surface libre.

Les données sont les suivantes :

Longueur : $L = 50.\,  \mathrm{m} $

Diamètre : $D = 65.\,  \mathrm{mm} $

Débit volumique : $Q = 11600.\,  \mathrm{L}\,  \mathrm{h}^{-1} $

Rugosité : $\epsilon = 230.\, \mu \mathrm{m} $

Hauteur de liquide $h = 8.0\,  \mathrm{m} $

Masse volumique du liquide $\rho = 900.\,  \mathrm{kg}\,  \mathrm{m}^{-3} $

Viscosité du liquide $\mu =  1.2 \, 10^{-3} \,  \mathrm{Pa}\,  \mathrm{s} $

 

Calculez :

 

Le nombre de Reynolds : $\text{Re} =  $
\begin{numerical}[points=1] 
\item[tolerance={2.3669196665e+03}] 4.7338393330e+04 
\end{numerical} 
 $\,$ 
 $ \quad (47338. \, - \, 79577.) $ 

La hauteur de pertes de charge : $h_v =  $
\begin{numerical}[points=1] 
\item[tolerance={5.4734239407e-02}] 1.0946847881e+00 
\end{numerical} 
 $\,  \mathrm{m}$ 
 $ \quad (1.0 \, - \, 5.4) $ 

La puissance nécessaire de la pompe : $\dot{W}_u =  $
\begin{numerical}[points=1] 
\item[tolerance={1.2936734377e+01}] 2.5873468754e+02 
\end{numerical} 
 $\,  \mathrm{W}$ 
 $ \quad (258. \, - \, 653.) $ 

(dans vos réponses, utilisez les notations scientifiques si besoin est. Par exemple $6.34\, 10^{-5}$ s'écrit 6.34e-5 et $10^{3}$ s'écrit 1e3).

Vous avez droit à une marge d'erreur relative de $5.0\, \% $

Des fourchettes indicatives vous sont fournies en face de chaque réponse. Ce ne sont que des ordres de grandeur pour vous inciter à vérifier vos calculs avant de valider vos réponses, et la bonne réponse peut sortir légèrement de l'intervalle indiqué.

\end{cloze} 


\end{quiz}
\end{document}
