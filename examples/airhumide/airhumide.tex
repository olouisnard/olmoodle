\documentclass[12pt]{article}

\usepackage[a4paper,landscape]{geometry}

\usepackage{moodle}

\usepackage{amsmath}
\usepackage{amssymb}
\usepackage{amsfonts}
\usepackage{amsthm,thmtools}
\usepackage{graphicx}

\usepackage{layouts}


\usepackage[utf8]{inputenc}
\usepackage[cyr]{aeguill}
\usepackage{xspace}


\usepackage[french]{babel}

\usepackage{ifthen}
\usepackage{verbatim} % Pour mettre des parties en commentaire

\usepackage{mymaths}
\usepackage{hyperref}



\begin{document}



\begin{quiz}{NUMERIQUES/AIRHUMIDE} 

 \begin{cloze}{Air humide} 
Le thermomètre sec d’une centrale météorologique indique une température $\theta = 24.\,  \mathrm{^\circ\mathrm{C}} $

Le capteur d'humidité indique une humidité relative $\psi = 58.\, \% $

La pression atmosphérique vaut $p_{\text{atm}} = 101325.\,  \mathrm{Pa} $

 

Calculez les grandeurs ci-dessous.

(dans vos réponses, utilisez les notations scientifiques si besoin est. Par exemple $6.34\, 10^{-5}$ s'écrit 6.34e-5 et $10^{3}$ s'écrit 1e3).

Vous avez droit à une marge d'erreur relative de $10.0\, \% $

Des fourchettes indicatives vous sont fournies en face de chaque réponse. Ce ne sont que des ordres de grandeur pour vous inciter à vérifier vos calculs avant de valider vos réponses, et la bonne réponse peut sortir légèrement de l'intervalle indiqué.

Humidité absolue : $w =  $
\begin{numerical}[points=1] 
\item[tolerance={1.0816299475e+00}] 1.0816299475e+01 
\end{numerical} 
 $\,  \mathrm{g}\,  \mathrm{kg}^{-1}$ 
 $ \quad (8. \, - \, 13.) $ 

Volume massique par unité de masse AS : $v =  $
\begin{numerical}[points=1] 
\item[tolerance={8.5641304330e-02}] 8.5641304330e-01 
\end{numerical} 
 $\,  \mathrm{m}^{3}\,  \mathrm{kg}^{-1}$ 
 $ \quad ( 8.30 \, 10^{-1}  \, - \,  8.75 \, 10^{-1} ) $ 

Enthalpie par unité de masse AS : $h =  $
\begin{numerical}[points=2] 
\item[tolerance={5.1647265839e+00}] 5.1647265839e+01 
\end{numerical} 
 $\,  \mathrm{kJ}\,  \mathrm{kg}^{-1}$ 
 $ \quad (36. \, - \, 63.) $ 

\end{cloze} 


 \begin{cloze}{Air humide} 
Le thermomètre sec d’une centrale météorologique indique une température $\theta = 25.\,  \mathrm{^\circ\mathrm{C}} $

Le capteur d'humidité indique une humidité relative $\psi = 66.\, \% $

La pression atmosphérique vaut $p_{\text{atm}} = 101325.\,  \mathrm{Pa} $

 

Calculez les grandeurs ci-dessous.

(dans vos réponses, utilisez les notations scientifiques si besoin est. Par exemple $6.34\, 10^{-5}$ s'écrit 6.34e-5 et $10^{3}$ s'écrit 1e3).

Vous avez droit à une marge d'erreur relative de $10.0\, \% $

Des fourchettes indicatives vous sont fournies en face de chaque réponse. Ce ne sont que des ordres de grandeur pour vous inciter à vérifier vos calculs avant de valider vos réponses, et la bonne réponse peut sortir légèrement de l'intervalle indiqué.

Humidité absolue : $w =  $
\begin{numerical}[points=1] 
\item[tolerance={1.3115033324e+00}] 1.3115033324e+01 
\end{numerical} 
 $\,  \mathrm{g}\,  \mathrm{kg}^{-1}$ 
 $ \quad (8. \, - \, 13.) $ 

Volume massique par unité de masse AS : $v =  $
\begin{numerical}[points=1] 
\item[tolerance={8.6241672355e-02}] 8.6241672355e-01 
\end{numerical} 
 $\,  \mathrm{m}^{3}\,  \mathrm{kg}^{-1}$ 
 $ \quad ( 8.30 \, 10^{-1}  \, - \,  8.75 \, 10^{-1} ) $ 

Enthalpie par unité de masse AS : $h =  $
\begin{numerical}[points=2] 
\item[tolerance={5.8526366869e+00}] 5.8526366869e+01 
\end{numerical} 
 $\,  \mathrm{kJ}\,  \mathrm{kg}^{-1}$ 
 $ \quad (36. \, - \, 63.) $ 

\end{cloze} 


 \begin{cloze}{Air humide} 
Le thermomètre sec d’une centrale météorologique indique une température $\theta = 24.\,  \mathrm{^\circ\mathrm{C}} $

Le capteur d'humidité indique une humidité relative $\psi = 70.\, \% $

La pression atmosphérique vaut $p_{\text{atm}} = 101325.\,  \mathrm{Pa} $

 

Calculez les grandeurs ci-dessous.

(dans vos réponses, utilisez les notations scientifiques si besoin est. Par exemple $6.34\, 10^{-5}$ s'écrit 6.34e-5 et $10^{3}$ s'écrit 1e3).

Vous avez droit à une marge d'erreur relative de $10.0\, \% $

Des fourchettes indicatives vous sont fournies en face de chaque réponse. Ce ne sont que des ordres de grandeur pour vous inciter à vérifier vos calculs avant de valider vos réponses, et la bonne réponse peut sortir légèrement de l'intervalle indiqué.

Humidité absolue : $w =  $
\begin{numerical}[points=1] 
\item[tolerance={1.3101293352e+00}] 1.3101293352e+01 
\end{numerical} 
 $\,  \mathrm{g}\,  \mathrm{kg}^{-1}$ 
 $ \quad (8. \, - \, 13.) $ 

Volume massique par unité de masse AS : $v =  $
\begin{numerical}[points=1] 
\item[tolerance={8.5950556792e-02}] 8.5950556792e-01 
\end{numerical} 
 $\,  \mathrm{m}^{3}\,  \mathrm{kg}^{-1}$ 
 $ \quad ( 8.30 \, 10^{-1}  \, - \,  8.75 \, 10^{-1} ) $ 

Enthalpie par unité de masse AS : $h =  $
\begin{numerical}[points=2] 
\item[tolerance={5.7462529556e+00}] 5.7462529556e+01 
\end{numerical} 
 $\,  \mathrm{kJ}\,  \mathrm{kg}^{-1}$ 
 $ \quad (36. \, - \, 63.) $ 

\end{cloze} 


 \begin{cloze}{Air humide} 
Le thermomètre sec d’une centrale météorologique indique une température $\theta = 29.\,  \mathrm{^\circ\mathrm{C}} $

Le capteur d'humidité indique une humidité relative $\psi = 54.\, \% $

La pression atmosphérique vaut $p_{\text{atm}} = 101325.\,  \mathrm{Pa} $

 

Calculez les grandeurs ci-dessous.

(dans vos réponses, utilisez les notations scientifiques si besoin est. Par exemple $6.34\, 10^{-5}$ s'écrit 6.34e-5 et $10^{3}$ s'écrit 1e3).

Vous avez droit à une marge d'erreur relative de $10.0\, \% $

Des fourchettes indicatives vous sont fournies en face de chaque réponse. Ce ne sont que des ordres de grandeur pour vous inciter à vérifier vos calculs avant de valider vos réponses, et la bonne réponse peut sortir légèrement de l'intervalle indiqué.

Humidité absolue : $w =  $
\begin{numerical}[points=1] 
\item[tolerance={1.3582319160e+00}] 1.3582319160e+01 
\end{numerical} 
 $\,  \mathrm{g}\,  \mathrm{kg}^{-1}$ 
 $ \quad (8. \, - \, 13.) $ 

Volume massique par unité de masse AS : $v =  $
\begin{numerical}[points=1] 
\item[tolerance={8.7463003205e-02}] 8.7463003205e-01 
\end{numerical} 
 $\,  \mathrm{m}^{3}\,  \mathrm{kg}^{-1}$ 
 $ \quad ( 8.30 \, 10^{-1}  \, - \,  8.75 \, 10^{-1} ) $ 

Enthalpie par unité de masse AS : $h =  $
\begin{numerical}[points=2] 
\item[tolerance={6.3835329721e+00}] 6.3835329721e+01 
\end{numerical} 
 $\,  \mathrm{kJ}\,  \mathrm{kg}^{-1}$ 
 $ \quad (36. \, - \, 63.) $ 

\end{cloze} 


 \begin{cloze}{Air humide} 
Le thermomètre sec d’une centrale météorologique indique une température $\theta = 18.\,  \mathrm{^\circ\mathrm{C}} $

Le capteur d'humidité indique une humidité relative $\psi = 66.\, \% $

La pression atmosphérique vaut $p_{\text{atm}} = 101325.\,  \mathrm{Pa} $

 

Calculez les grandeurs ci-dessous.

(dans vos réponses, utilisez les notations scientifiques si besoin est. Par exemple $6.34\, 10^{-5}$ s'écrit 6.34e-5 et $10^{3}$ s'écrit 1e3).

Vous avez droit à une marge d'erreur relative de $10.0\, \% $

Des fourchettes indicatives vous sont fournies en face de chaque réponse. Ce ne sont que des ordres de grandeur pour vous inciter à vérifier vos calculs avant de valider vos réponses, et la bonne réponse peut sortir légèrement de l'intervalle indiqué.

Humidité absolue : $w =  $
\begin{numerical}[points=1] 
\item[tolerance={8.4780040261e-01}] 8.4780040261e+00 
\end{numerical} 
 $\,  \mathrm{g}\,  \mathrm{kg}^{-1}$ 
 $ \quad (8. \, - \, 13.) $ 

Volume massique par unité de masse AS : $v =  $
\begin{numerical}[points=1] 
\item[tolerance={8.3601974032e-02}] 8.3601974032e-01 
\end{numerical} 
 $\,  \mathrm{m}^{3}\,  \mathrm{kg}^{-1}$ 
 $ \quad ( 8.30 \, 10^{-1}  \, - \,  8.75 \, 10^{-1} ) $ 

Enthalpie par unité de masse AS : $h =  $
\begin{numerical}[points=2] 
\item[tolerance={3.9573770882e+00}] 3.9573770882e+01 
\end{numerical} 
 $\,  \mathrm{kJ}\,  \mathrm{kg}^{-1}$ 
 $ \quad (36. \, - \, 63.) $ 

\end{cloze} 


 \begin{cloze}{Air humide} 
Le thermomètre sec d’une centrale météorologique indique une température $\theta = 25.\,  \mathrm{^\circ\mathrm{C}} $

Le capteur d'humidité indique une humidité relative $\psi = 56.\, \% $

La pression atmosphérique vaut $p_{\text{atm}} = 101325.\,  \mathrm{Pa} $

 

Calculez les grandeurs ci-dessous.

(dans vos réponses, utilisez les notations scientifiques si besoin est. Par exemple $6.34\, 10^{-5}$ s'écrit 6.34e-5 et $10^{3}$ s'écrit 1e3).

Vous avez droit à une marge d'erreur relative de $10.0\, \% $

Des fourchettes indicatives vous sont fournies en face de chaque réponse. Ce ne sont que des ordres de grandeur pour vous inciter à vérifier vos calculs avant de valider vos réponses, et la bonne réponse peut sortir légèrement de l'intervalle indiqué.

Humidité absolue : $w =  $
\begin{numerical}[points=1] 
\item[tolerance={1.1092467680e+00}] 1.1092467680e+01 
\end{numerical} 
 $\,  \mathrm{g}\,  \mathrm{kg}^{-1}$ 
 $ \quad (8. \, - \, 13.) $ 

Volume massique par unité de masse AS : $v =  $
\begin{numerical}[points=1] 
\item[tolerance={8.5967015887e-02}] 8.5967015887e-01 
\end{numerical} 
 $\,  \mathrm{m}^{3}\,  \mathrm{kg}^{-1}$ 
 $ \quad ( 8.30 \, 10^{-1}  \, - \,  8.75 \, 10^{-1} ) $ 

Enthalpie par unité de masse AS : $h =  $
\begin{numerical}[points=2] 
\item[tolerance={5.3375296687e+00}] 5.3375296687e+01 
\end{numerical} 
 $\,  \mathrm{kJ}\,  \mathrm{kg}^{-1}$ 
 $ \quad (36. \, - \, 63.) $ 

\end{cloze} 


 \begin{cloze}{Air humide} 
Le thermomètre sec d’une centrale météorologique indique une température $\theta = 18.\,  \mathrm{^\circ\mathrm{C}} $

Le capteur d'humidité indique une humidité relative $\psi = 73.\, \% $

La pression atmosphérique vaut $p_{\text{atm}} = 101325.\,  \mathrm{Pa} $

 

Calculez les grandeurs ci-dessous.

(dans vos réponses, utilisez les notations scientifiques si besoin est. Par exemple $6.34\, 10^{-5}$ s'écrit 6.34e-5 et $10^{3}$ s'écrit 1e3).

Vous avez droit à une marge d'erreur relative de $10.0\, \% $

Des fourchettes indicatives vous sont fournies en face de chaque réponse. Ce ne sont que des ordres de grandeur pour vous inciter à vérifier vos calculs avant de valider vos réponses, et la bonne réponse peut sortir légèrement de l'intervalle indiqué.

Humidité absolue : $w =  $
\begin{numerical}[points=1] 
\item[tolerance={9.3907625611e-01}] 9.3907625611e+00 
\end{numerical} 
 $\,  \mathrm{g}\,  \mathrm{kg}^{-1}$ 
 $ \quad (8. \, - \, 13.) $ 

Volume massique par unité de masse AS : $v =  $
\begin{numerical}[points=1] 
\item[tolerance={8.3723012966e-02}] 8.3723012966e-01 
\end{numerical} 
 $\,  \mathrm{m}^{3}\,  \mathrm{kg}^{-1}$ 
 $ \quad ( 8.30 \, 10^{-1}  \, - \,  8.75 \, 10^{-1} ) $ 

Enthalpie par unité de masse AS : $h =  $
\begin{numerical}[points=2] 
\item[tolerance={4.1886755776e+00}] 4.1886755776e+01 
\end{numerical} 
 $\,  \mathrm{kJ}\,  \mathrm{kg}^{-1}$ 
 $ \quad (36. \, - \, 63.) $ 

\end{cloze} 


 \begin{cloze}{Air humide} 
Le thermomètre sec d’une centrale météorologique indique une température $\theta = 16.\,  \mathrm{^\circ\mathrm{C}} $

Le capteur d'humidité indique une humidité relative $\psi = 73.\, \% $

La pression atmosphérique vaut $p_{\text{atm}} = 101325.\,  \mathrm{Pa} $

 

Calculez les grandeurs ci-dessous.

(dans vos réponses, utilisez les notations scientifiques si besoin est. Par exemple $6.34\, 10^{-5}$ s'écrit 6.34e-5 et $10^{3}$ s'écrit 1e3).

Vous avez droit à une marge d'erreur relative de $10.0\, \% $

Des fourchettes indicatives vous sont fournies en face de chaque réponse. Ce ne sont que des ordres de grandeur pour vous inciter à vérifier vos calculs avant de valider vos réponses, et la bonne réponse peut sortir légèrement de l'intervalle indiqué.

Humidité absolue : $w =  $
\begin{numerical}[points=1] 
\item[tolerance={8.2565452008e-01}] 8.2565452008e+00 
\end{numerical} 
 $\,  \mathrm{g}\,  \mathrm{kg}^{-1}$ 
 $ \quad (8. \, - \, 13.) $ 

Volume massique par unité de masse AS : $v =  $
\begin{numerical}[points=1] 
\item[tolerance={8.2998520600e-02}] 8.2998520600e-01 
\end{numerical} 
 $\,  \mathrm{m}^{3}\,  \mathrm{kg}^{-1}$ 
 $ \quad ( 8.30 \, 10^{-1}  \, - \,  8.75 \, 10^{-1} ) $ 

Enthalpie par unité de masse AS : $h =  $
\begin{numerical}[points=2] 
\item[tolerance={3.6972527107e+00}] 3.6972527107e+01 
\end{numerical} 
 $\,  \mathrm{kJ}\,  \mathrm{kg}^{-1}$ 
 $ \quad (36. \, - \, 63.) $ 

\end{cloze} 


 \begin{cloze}{Air humide} 
Le thermomètre sec d’une centrale météorologique indique une température $\theta = 24.\,  \mathrm{^\circ\mathrm{C}} $

Le capteur d'humidité indique une humidité relative $\psi = 50.\, \% $

La pression atmosphérique vaut $p_{\text{atm}} = 101325.\,  \mathrm{Pa} $

 

Calculez les grandeurs ci-dessous.

(dans vos réponses, utilisez les notations scientifiques si besoin est. Par exemple $6.34\, 10^{-5}$ s'écrit 6.34e-5 et $10^{3}$ s'écrit 1e3).

Vous avez droit à une marge d'erreur relative de $10.0\, \% $

Des fourchettes indicatives vous sont fournies en face de chaque réponse. Ce ne sont que des ordres de grandeur pour vous inciter à vérifier vos calculs avant de valider vos réponses, et la bonne réponse peut sortir légèrement de l'intervalle indiqué.

Humidité absolue : $w =  $
\begin{numerical}[points=1] 
\item[tolerance={9.3020833320e-01}] 9.3020833320e+00 
\end{numerical} 
 $\,  \mathrm{g}\,  \mathrm{kg}^{-1}$ 
 $ \quad (8. \, - \, 13.) $ 

Volume massique par unité de masse AS : $v =  $
\begin{numerical}[points=1] 
\item[tolerance={8.5436369395e-02}] 8.5436369395e-01 
\end{numerical} 
 $\,  \mathrm{m}^{3}\,  \mathrm{kg}^{-1}$ 
 $ \quad ( 8.30 \, 10^{-1}  \, - \,  8.75 \, 10^{-1} ) $ 

Enthalpie par unité de masse AS : $h =  $
\begin{numerical}[points=2] 
\item[tolerance={4.7793616038e+00}] 4.7793616038e+01 
\end{numerical} 
 $\,  \mathrm{kJ}\,  \mathrm{kg}^{-1}$ 
 $ \quad (36. \, - \, 63.) $ 

\end{cloze} 


 \begin{cloze}{Air humide} 
Le thermomètre sec d’une centrale météorologique indique une température $\theta = 21.\,  \mathrm{^\circ\mathrm{C}} $

Le capteur d'humidité indique une humidité relative $\psi = 64.\, \% $

La pression atmosphérique vaut $p_{\text{atm}} = 101325.\,  \mathrm{Pa} $

 

Calculez les grandeurs ci-dessous.

(dans vos réponses, utilisez les notations scientifiques si besoin est. Par exemple $6.34\, 10^{-5}$ s'écrit 6.34e-5 et $10^{3}$ s'écrit 1e3).

Vous avez droit à une marge d'erreur relative de $10.0\, \% $

Des fourchettes indicatives vous sont fournies en face de chaque réponse. Ce ne sont que des ordres de grandeur pour vous inciter à vérifier vos calculs avant de valider vos réponses, et la bonne réponse peut sortir légèrement de l'intervalle indiqué.

Humidité absolue : $w =  $
\begin{numerical}[points=1] 
\item[tolerance={9.9313855729e-01}] 9.9313855729e+00 
\end{numerical} 
 $\,  \mathrm{g}\,  \mathrm{kg}^{-1}$ 
 $ \quad (8. \, - \, 13.) $ 

Volume massique par unité de masse AS : $v =  $
\begin{numerical}[points=1] 
\item[tolerance={8.4658121678e-02}] 8.4658121678e-01 
\end{numerical} 
 $\,  \mathrm{m}^{3}\,  \mathrm{kg}^{-1}$ 
 $ \quad ( 8.30 \, 10^{-1}  \, - \,  8.75 \, 10^{-1} ) $ 

Enthalpie par unité de masse AS : $h =  $
\begin{numerical}[points=2] 
\item[tolerance={4.6325952290e+00}] 4.6325952290e+01 
\end{numerical} 
 $\,  \mathrm{kJ}\,  \mathrm{kg}^{-1}$ 
 $ \quad (36. \, - \, 63.) $ 

\end{cloze} 


\end{quiz}
\end{document}
